\documentclass[12pt,a4paper,oneside]{book}
%-------------------------------------------------------------------
\usepackage[utf8]{inputenc}
\usepackage{vietnam}
\usepackage{amsfonts}
\usepackage{latexsym, amsmath, amsxtra, amssymb, amscd, amsthm,fancyhdr} %, txfonts, pxfonts}
\usepackage[top=2.5cm, bottom=2.5cm, left=2.7cm, right=2cm]{geometry}
\usepackage{indentfirst} %thụt lề dòng đầu tiên
\usepackage{graphicx} % chèn ảnh,
\usepackage{algorithm} %1 chèn thuật toán
\usepackage{algpseudocode} %2 chèn thuật toán
\setcounter{secnumdepth}{4} % subsubsubsection
\usepackage{placeins} % để sử dụng \FloatBarrier cố định vị trí
\usepackage{subfigure} % chèn hình phụ
\usepackage{pdfpages} % chèn pdf vào luận văn
\usepackage{tabularx} % chèn bảng
\usepackage{multirow} % bảng có thể gộp hàng hoặc cột
\usepackage{hyperref} % cite papers, trinh tu build f6 f11 f6 f1
\usepackage{pgf,tikz,pgfplots} % hình từ geogebra
\usetikzlibrary{arrows} % hình từ geogebra
\pgfplotsset{compat=1.5} % hình từ geogebra gốc là 1.15
\usepackage{mathrsfs} % hình từ geogebra
\usepackage{fancyhdr} % header và footer
\usepackage{caption} % không hiện 1 table or figure in list of t, f
\usepackage{makecell} % xuong hang trong 1 cell table

% Set up------------------------------------------------------------
\renewcommand{\baselinestretch}{1.5} %khoảng cách giữa các dòng
\setlength{\parskip}{0.5em} %khoảng cách giữa các đoạn
\numberwithin{equation}{chapter} % đánh số công thức theo chapter
\numberwithin{figure}{chapter} % đánh số hình theo chapter
\numberwithin{table}{chapter} % đánh số bảng theo chapter
\renewcommand{\algorithmicrequire}{\textbf{Input:}} %thay require trong algorithmic thanh Input
\renewcommand{\algorithmicensure}{\textbf{Output:}} %thay ensure trong algorithmic thanh Output

\pagestyle{fancy}
\fancyhf{}
\rhead{\fontsize{10pt}{0pt}{\selectfont{\leftmark}}}
\rfoot{\fontsize{10pt}{0pt}{\selectfont{SVTH: Nguyễn Thành Đạt}}}
\cfoot{\thepage}
%\rfoot{\fontsize{10pt}{0pt}{\selectfont{GVHD: PSG.TS Hà Hoàng Kha}}}
\renewcommand{\headrulewidth}{0.5pt}
\renewcommand{\footrulewidth}{0.5pt}

\fancypagestyle{loi_cam_doan}{\fancyhf{}
\rhead{\fontsize{10pt}{0pt}{\selectfont{LỜI CAM ĐOAN}}}
\rfoot{\fontsize{10pt}{0pt}{\selectfont{SVTH: Nguyễn Thành Đạt}}}
\cfoot{\thepage}
%\rfoot{\fontsize{10pt}{0pt}{\selectfont{GVHD: PGS.TS Hà Hoàng Kha}}}
\renewcommand{\headrulewidth}{0.5pt}
\renewcommand{\footrulewidth}{0.5pt}
}

\fancypagestyle{loi_cam_on}{\fancyhf{}
\rhead{\fontsize{10pt}{0pt}{\selectfont{LỜI CẢM ƠN}}}
\rfoot{\fontsize{10pt}{0pt}{\selectfont{SVTH: Nguyễn Thành Đạt}}}
\cfoot{\thepage}
%\rfoot{\fontsize{10pt}{0pt}{\selectfont{GVHD: PGS.TS Hà Hoàng Kha}}}
\renewcommand{\headrulewidth}{0.5pt}
\renewcommand{\footrulewidth}{0.5pt}
}

\fancypagestyle{tom_tat}{\fancyhf{}
\rhead{\fontsize{10pt}{0pt}{\selectfont{TÓM TẮT}}}
\rfoot{\fontsize{10pt}{0pt}{\selectfont{SVTH: Nguyễn Thành Đạt}}}
\cfoot{\thepage}
%\rfoot{\fontsize{10pt}{0pt}{\selectfont{GVHD: PGS.TS Hà Hoàng Kha}}}
\renewcommand{\headrulewidth}{0.5pt}
\renewcommand{\footrulewidth}{0.5pt}
}

\fancypagestyle{abstract}{\fancyhf{} %
\rhead{\fontsize{10pt}{0pt}{\selectfont{ABSTRACT}}}%
\rfoot{\fontsize{10pt}{0pt}{\selectfont{SVTH: Nguyễn Thành Đạt}}}
\cfoot{\thepage}
%\rfoot{\fontsize{10pt}{0pt}{\selectfont{GVHD: PGS.TS Hà Hoàng Kha}}}
\renewcommand{\headrulewidth}{0.5pt}
\renewcommand{\footrulewidth}{0.5pt}
}

\fancypagestyle{danhmucviettat}{\fancyhf{}
\rhead{\fontsize{10pt}{0pt}{\selectfont{DANH MỤC VIẾT TẮT}}}
\rfoot{\fontsize{10pt}{0pt}{\selectfont{SVTH: Nguyễn Thành Đạt}}}
\cfoot{\thepage}
%\rfoot{\fontsize{10pt}{0pt}{\selectfont{GVHD: PGS.TS Hà Hoàng Kha}}}
\renewcommand{\headrulewidth}{0.5pt}
\renewcommand{\footrulewidth}{0.5pt}
}

\fancypagestyle{phuluc}{\fancyhf{}
\rhead{\fontsize{10pt}{0pt}{\selectfont{PHỤ LỤC}}}
\lfoot{\fontsize{10pt}{0pt}{\selectfont{SVTH: Nguyễn Thành Đạt}}}
\cfoot{\thepage}
%\rfoot{\fontsize{10pt}{0pt}{\selectfont{GVHD: PGS.TS Hà Hoàng Kha}}}
\renewcommand{\headrulewidth}{0.5pt}
\renewcommand{\footrulewidth}{0.5pt}
}

\begin{document}
%\fontsize{13pt}{13pt}\selectfont
\newcolumntype{L}[1]{>{\raggedright\arraybackslash}p{#1}} % có thể định nghĩa kich thước cho bảng
\newcolumntype{C}[1]{>{\centering\arraybackslash}p{#1}}
\newcolumntype{R}[1]{>{\raggedleft\arraybackslash}p{#1}}

\begin{titlepage}

\newcommand{\HRule}{\rule{\linewidth}{0.5mm}}
% \renewcommand{\baselinestretch}{2.0}
\centering

\large ĐẠI HỌC QUỐC GIA THÀNH PHỐ HỒ CHÍ MINH \\
TRƯỜNG ĐẠI HỌC BÁCH KHOA \\
KHOA ĐIỆN - ĐIỆN TỬ  \\
BỘ MÔN VIỄN THÔNG  \\ [0.8cm]

\begin{figure}[!ht]
    \centering
    \includegraphics[scale=0.3]{front_pages/Logo_BK.png}
    \label{fig:logo}
\end{figure}

\bfseries
\Large{LUẬN VĂN TỐT NGHIỆP ĐẠI HỌC} \\[1.3cm]

\LARGE{NHẬN DẠNG NGÔN NGỮ KÝ HIỆU CHO NGƯỜI KHIẾM THÍNH SỬ DỤNG KỸ THUẬT HỌC SÂU: TÁCH VÀ PHÂN TÍCH ĐẶC TRƯNG KHUNG XƯƠNG TRÊN VIDEO RGB} \\ [2.1cm]

\normalfont
\normalsize
% \small
\begin{tabular}{r l}
    \fontsize{14pt}{0pt}{\selectfont{GVHD:}} & \fontsize{14pt}{0pt}{\selectfont{PGS.TS HÀ HOÀNG KHA}} \\
    \fontsize{14pt}{0pt}{\selectfont{SVTH:}} & \fontsize{14pt}{0pt}{\selectfont{NGUYỄN THÀNH ĐẠT - 1510698}}  					
                        
\end{tabular} 
\\ [2.1cm]

TP. HỒ CHÍ MINH, THÁNG 12 NĂM 2019
% \vfill
\end{titlepage}


%------------------------------------------------------------------
\includepdf[pages=-]{front_pages/nhiem_vu.pdf}

%------------------------------------------------------------------
\newpage
\pagenumbering{roman}
\addcontentsline{toc}{chapter}{LỜI CẢM ƠN}
\thispagestyle{loi_cam_on}
\begin{center}
{\fontsize{18pt}{30pt}{\selectfont{\textbf{Lời cảm ơn}}}}
\end{center}

\textit{Trong thời gian thực hiện luận án này, em đã nhận được sự hỗ trợ nhiệt tình, hướng dẫn tận tình và những lời động viên tích cực từ các giảng viên, bạn bè và gia đình. Do đó, em đã hoàn thành luận án như mục tiêu đã đặt ra. Những lời giảng dạy quý báu, không những về mặt kiến thức mà còn về đạo đức làm người của các quý thầy cô sẽ là hành trang cho con đường tương lai của các thế hệ sinh viên.}

\textit{Em xin gửi đến thầy Hà Hoàng Kha, giảng viên hướng dẫn trực tiếp đề tài, lời biết ơn sâu sắc. Thầy đã dành thời gian quý báu để gặp gỡ, thảo luận, đưa ra những vấn đề hay để rèn luyện chúng em khả năng giải quyết vấn đề. Thầy là người đã theo tường bước đi của chúng tôi, tận tình chỉ bảo, hướng dẫn chúng tôi từ khi làm đề cương, thực tập cho đến luận văn. Ngoài những kiến thức chuyên ngành, chúng tôi còn nhận được những lời khuyên, kinh nghiệm quý giá trong học tập và nghiên cứu từ thầy.}

\textit{Em cũng xin cảm ơn chân thành đến cha mẹ đã động viên và tạo điều kiện giúp đỡ chúng tôi vượt qua những khó khăn trong suốt quá trình học tập và nghiên cứu.}

\textit{Mặc dù đã cố gắng trong phạm vi và khả năng cho phép, nhưng luận văn không thể tránh khỏi những thiếu sót, rất mong được sự góp ý của quý thầy cô và các bạn.}

\textit{Cuối cùng, xin chân thành cảm ơn quý thầy cô và các bạn đã dành thời gian đọc luận văn này.}


%------------------------------------------------------------------
\newpage
\addcontentsline{toc}{chapter}{LỜI CAM ĐOAN}
\thispagestyle{loi_cam_doan}
\begin{center}
{\fontsize{18pt}{30pt}{\selectfont{\textbf{Lời cam đoan}}}}
\end{center}

\noindent Tôi tên: Nguyễn Thành Đạt là sinh viên chuyên ngành Điện tử - Viễn thông khóa 2015. Tôi xin cam đoan rằng, luận văn tốt nghiệp "Nhận dạng ngôn ngữ ký hiệu cho người khiếm thính sử dụng kỹ thuật học sâu: tách và phân tích đặc trưng khung xương trên video RGB" là công trình nghiên cứu của em dưới sự hướng dẫn của PGS.TS Hà Hoàng Kha. Đề tài xuất phát từ nhu cầu thực tiễn và nguyện vọng muốn thực hiện của em. Ngoại trừ những nội dung được biên soạn lại từ các công trình khác đã ghi rõ, các nội dung, kết quả kiểm tra đánh giá thực nghiệm trình bày trong luận văn này là kết quả nghiên cứu do chính em thực hiện, hoàn toàn không phải sao chép từ bất kỳ một tài liệu hoặc công trình nghiên cứu nào khác.

Nếu không thực hiện đúng các cam kết trên, em xin hoàn toàn chịu trách nhiệm trước kỷ luật của nhà trường cũng như pháp luật Nhà nước.

\begin{flushright}
Sinh viên thực hiện
\end{flushright}


%------------------------------------------------------------------
\newpage
\addcontentsline{toc}{chapter}{TÓM TẮT}
\thispagestyle{tom_tat}
\begin{center}
{\fontsize{18pt}{30pt}{\selectfont{\textbf{Tóm tắt}}}}
\end{center}

Trong luận văn này, em đã thực hiện viết ứng dụng realtime nhận diện được một số từ ngữ thông dụng trong ngôn ngữ ký hiệu của người khiếm thính khi giao tiếp. Ứng dụng được xây dựng với mục đích giúp cho người khiếm thính giao tiếp dễ dàng hơn với người bình thường khi họ không hiếu ngôn ngữ ký hiệu của người khiếm thính.

Bởi những phát triển trong học máy và học sâu gần đây, đã giúp con người giải quyết được những bài toán thực tế mà trước đây tưởng chừng như máy tính không thể làm được. Trong ứng dụng này em đã sử dụng một trong các phương pháp học máy đó để giải quyết bài toán nhận diện ngôn ngữ ký hiệu cho người khiếm thính Việt Nam. Phương pháp nhận diện mà đề tài sử dụng chia làm 2 bước. Đầu tiên từ hình ảnh RGB có chứa hình ảnh người khiếm thính đang diễn tả một từ ngữ do camera truyền vào. Ứng dụng sử dụng một mạng CNN là mạng mobilenet để trích xuất đặc trưng khung xương là tọa độ các khớp xương trên ảnh 2D (tổng cộng có 18 khớp xương ban đầu). Sau đó các tọa độ khớp xương này được đưa vào một mạng DNN để từ đó phân loại ra các hành động diễn tả từ ngữ mà mạng đã học được. Cuối cùng ứng dụng xuất ra từ ngữ mà người đứng trước camera muốn diễn tả và xuất ra màn hình.

Phương pháp nhận diện dựa vào trích xuất đặc trưng khung xương là một phương pháp khá hay và lạ so với các nghiên cứu liên quan trước đây. Điểm hay của phương pháp này là việc trích xuất khung xương đã khái quát gần như toàn bộ tư thế và hành động mà người khiếm thị muốn diễn tả. Việc này còn làm giảm số chiều dữ liệu so với việc từ một hình ảnh RGB đưa vào để phân loại ra từ ngữ gì. Việc còn lại chỉ cần phân loại hành động dựa trên đặc trưng đã được trích xuất bằng một mạng DNN cấu trúc nhỏ. Việc này giúp tăng tốc độ xử lý của ứng dụng lên đáp ứng được việc xử lý realtime.

Các kết quả thực nghiệm thu được từ hệ thống cho thấy tỷ lệ nhận dạng cao và tốc độ đáp ứng đủ nhanh cho hoạt động chế độ thời gian thực.

%------------------------------------------------------------------
\newpage
\thispagestyle{abstract}
\begin{center}
{\fontsize{18pt}{30pt}{\selectfont{\textbf{ABSTRACT}}}}
\end{center}
%\textbf{Key words:} 
In this thesis, I made writing realtime application that identifies some common words in sign language of deaf people when communicating. The application was built with the purpose of helping the hearing impaired to communicate more easily with ordinary people when they are not fond of the sign language of the hearing impaired.

Because of recent developments in machine learning and deep learning, it has helped people solve real-world problems that previously seemed impossible to computers. In this application, I used one of the machine learning methods to solve the sign language recognition problem for Vietnamese deaf people. The identification method that the subject uses is divided into 2 steps. First from the RGB image contains the image of the hearing impaired expressing a word transmitted by the camera. The application uses a CNN network, the mobilenet network, to extract the skeletal characteristic that is the coordinates of the joints in the 2D image (a total of 18 initial joints). These articular coordinates are then fed into a DNN network from which to categorize the words that the network has learned. Finally, the application outputs the words that the person in front of the camera wants to describe and outputs it to the screen.

The identification method based on skeletal feature extraction is a rather interesting and strange method compared to previous related studies. The beauty of this method is that the extraction of the skeleton has generalized almost the entire posture and action that the blind person wants to describe. This also reduces the number of data dimensions compared to what an RGB image is inserted to sort out. The rest just needs to classify the action based on the feature extracted by a small structured DNN network. This increases the speed of the application to meet realtime processing.

The experimental results obtained from the system show that the recognition rate is high and the response speed is fast enough for real time mode operation.

%------------------------------------------------------------------
\newpage
\addcontentsline{toc}{chapter}{MỤC LỤC}
{\fontsize{12pt}{5pt}\selectfont
\tableofcontents}


%------------------------------------------------------------------
\newpage
\addcontentsline{toc}{chapter}{DANH SÁCH HÌNH VẼ}
\listoffigures


%------------------------------------------------------------------
\newpage
\addcontentsline{toc}{chapter}{DANH SÁCH BẢNG}
\listoftables


%------------------------------------------------------------------
\newpage
\thispagestyle{danhmucviettat}
\addcontentsline{toc}{chapter}{DANH MỤC TỪ VIẾT TẮT}
{\fontsize{18pt}{30pt}{\selectfont{\textbf{Danh mục từ viết tắt}}}}


%CFMs & : Confidence Maps

\FloatBarrier
\begin{table}[h]
%\caption{Tổng quan các chương của luận văn}
\label{table:bo_cuc_luan_van}
\centering
\begin{center}
\begin{tabular}{|c|l|p{7cm}|} 
\hline 
NN & Neural Network & Mạng thần kinh\\ 
\hline
CNN & Convolutional Neural Networks & Mạng neuron tích chập\\
\hline 
DNN & Deep Neural Network & Mạng thần kinh sâu\\
\hline
HMM & Hidden Markov Model & Mô hình Markov ẩn \\
\hline 
DL & Deep Learning & Học sâu\\
\hline
MLP & Multilayer Perceptron & Perceptron đa tầng\\
\hline
GD & Gradient Descent & Giảm theo gradient\\
\hline
SGD & Stochastic Gradient Descent & Giảm theo gradient ngẫu nhiên\\
\hline
SJM & Skeleton Joints Mapping & Vector đặc trưng mang hình dáng của khung xương con người\\
\hline
PAFs & Part Affinity Fields &  Trường ái lực một phần (tạm dịch)\\
\hline
CFMs & Confidence Maps & Bản đồ tự tin(tạm dịch)\\
\hline
\end{tabular}
\end{center}
\end{table}
\FloatBarrier
 


\newpage
\setcounter{page}{1}
\pagenumbering{arabic}
\chapter{TỔNG QUAN}
\label{s:tong_quan}
Nội dung chương đặt vấn đề tổng quan bài toán nhận dạng ngôn ngữ ký hiệu hiện nay và một số các công trình nghiên cứu về nhận dạng ngôn ngữ ký hiệu đã công bố trong nước và quốc tế. Từ đó đưa ra lý do chọn đề tài nhận dạng ngôn ngữ ký hiệu, giới thiệu về công cụ hỗ trợ cũng như mục tiêu, nhiệm vụ của đề tài cần tập trung giải quyết. Ngoài ra, chương mở đầu cũng giới thiệu phương pháp xử lý đề tài luận văn và trình bày bố cục nội dung trình bày xuyên suốt bài báo cáo.

\section{Đặt vấn đề}
Khiếm thính là trình trạng một người có thính giác kém, không nghe được những âm thanh, tiếng nói mà một người bình thường có thể nghe được. Theo số liệu thống kê năm 2014 của Trung tâm nghiên cứu Giáo dục đặc biệt (Viện Khoa học GD Việt Nam), Việt Nam có khoảng 7 triệu người khuyết tật, trong đó có hơn 1 triệu người khiếm thính. Do khả năng nghe bị suy giảm nên việc giao tiếp bằng lời nói ở cộng đồng người khiếm thính  và với người bình thường dường như là không thể. Để thay thế cho việc giao tiếp bằng tiếng nói, “ngôn ngữ ký hiệu” được ra đời nhằm phục vụ việc giao tiếp trực tiếp mà không cần thông qua lời nói.
Ngôn ngữ ký hiệu hay ngôn ngữ dấu hiệu, thủ ngữ là ngôn ngữ dùng những biểu hiện của bàn tay thay cho âm thanh của tiếng nói. Ngôn ngữ ký hiệu do người khiếm thính tạo ra nhằm giúp họ có thể giao tiếp với nhau trong cộng đồng của mình và tiếp thu tri thức của xã hội. Ngôn ngữ ký hiệu không như chữ viết tay hay lời nói có một cách thức và ngữ pháp cụ thể, ký hiệu rõ ràng để có thể mô hình hóa được. Để sử dụng ngôn ngữ ký hiệu, người giao tiếp cần thể hiện cử chỉ bằng cả bàn tay và cánh tay, kết hợp với điệu bộ của cơ thể để có thể diễn tả ý nghĩa mong muốn. Tuy được cộng đồng người khiếm thính sử dụng phổ biến nhưng đối với người bình thường, hầu như đa số đều không hiểu ngôn ngữ ký hiệu. Ngoài ra, ngôn ngữ ký hiệu giữa các nước trên thế giới, thậm chí giữa các vùng miền trong một nước cũng có sự khác nhau. Việc này khiến cho việc giao tiếp của người khiếm thính gặp rất nhiều khó khăn. Vì vậy một hệ thống nhận dạng ngôn ngữ ký hiệu tự động phiên dịch sang tiếng nói giúp người khiếm thính hòa nhập cộng đồng là thật sự cần thiết.
Với những lý do đó, trong đề cương này em xin trình bày một mô hình nhận dạng ngôn ngữ ký hiệu qua phân tích hình ảnh từ camera. Hệ thống này có khả năng quan sát, phát hiện con người và nhận diện hành động mà người đứng trước camera muốn diễn đạt. 

\section{Những nghiên cứu liên quan}
\label{ss:nghien_cuu_lien_quan}
Lĩnh vực xử lý, phân loại, nhận diện ngôn ngữ ký hiệu rất rộng và phức tạp do tính phong phú của chữ cái, câu từ và đặc tính không cấu trúc của các hành động con người. Các hướng phát triển trong lĩnh vực này rất tiềm năng và bùng nổ trong những năm gần đây.

Trên thế giới, đã có nhiều nghiên cứu phát triển các dịch vụ thông dịch ngôn ngữ ký hiệu và các sản phẩm công nghệ nhằm hỗ trợ người khiếm thính trong giao tiếp xã hội. Một số sản phẩm nổi bật như găng tay chuyển đổi ngôn ngữ ký hiệu thành giọng nói \cite{tl1}, các phần mềm dịch từ văn bản/ giọng nói sang ngôn ngữ ký hiệu hay các từ điển tra cứu ngôn ngữ ký hiệu online \cite{tl2}. Một số tác giả cũng đã nghiên cứu sử dụng thiết bị Kinect trong việc nhận dạng các con số và các ký tự chữ cái theo ký hiệu ngôn ngữ người câm \cite{tl3}, tuy nhiên việc nhận dạng là dựa trên ảnh tĩnh chưa có những giải pháp nhận dạng ảnh động theo như các ký hiệu hiệu ngôn ngữ tiếng Việt.

Ở các nước, các nhà nghiên cứu đã tiếp cận bài toán nhận dạng cử chỉ bàn tay theo rất nhiều hướng khác nhau như dựa vào màu sắc bàn tay, hình dáng bàn tay hay công trình của Viola $\&$ Jones dùng các đặc trưng Haarlike.

Ngoài ra, lĩnh vực nhận diện cử chỉ của con người được quan tâm và nghiên cứu trong rất nhiều năm từ năm 1992 \cite{Yamato} với giải thuật HMM của Junji YAMATO cho đến những năm gần đây, sử dụng phương pháp SVM cục bộ của Christian Schudlt vào năm 2004 \cite{Schuldt:2004:RHA:1018429.1020906}, và các phương pháp khác như: giải thuật khung xương \cite{Chen:2006:HAR:1178782.1178808}, \cite{Forsyth}, \cite{With}. Các bài khảo sát chi tiết có thể được tìm thấy ở \cite{Cristani201386} khảo sát cách thức nhận diện hành động con người và đưa ra chi tiết nhiều bài báo có thể tham khảo.

Như đã đề cập ở các phần trên, phương pháp SVM chỉ hoạt động tốt với các ảnh tĩnh, nên các nhận diện chính xác của SVM phải nói chính xác là các cử chỉ của con người tại thời điểm tức thời đó đã được nhận diện chính xác, không phải là một hành động gồm nhiều cử chỉ. Luận văn này bên cạnh SVM thì cũng tập trung nhiều vào HMM nên các bài báo liên quan đến SVM và HMM đều được xem xét kỹ, trong đó có một số phương pháp rất hay và có thể được xem xét:

\begin{itemize}
\item Xây dựng mô hình 3D dựa vào nhiều camera ở các vị trí khác nhau \cite{HLUT1}
\item Nhận diện hành động con người dựa trên mô tả các đặc trưng góc 3D \cite{6693448}
\item Sử dụng nhiều thiết bị được gắn trên người \cite{4650859}
\end{itemize}

Mặc dù đã được nghiên cứu trong thời gian dài, tuy nhiên do đặc điểm của việc nhận dạng hành động con người rất phức tạp và không có một cấu trúc rõ ràng dẫn đến việc rất khó áp dụng cho thực tế hoặc một lĩnh vực rộng rãi.

\section{Mục tiêu của luận văn}

Để có thể thực hiện việc nhận dạng ngôn ngữ ký hiệu, cần xác định được phương pháp thực hiện, lựa chọn các giải thuật hợp lý phù hợp với điều kiện thực tế và khả năng có thể ứng dụng cao nhất.
Phương án thực hiện được lựa chọn ban đầu có 3 phương án: 

Phương án 1: sử dụng thiết bị kinect để trích xuất hình dáng khung xương sau đó nhận dạng ngôn ngữ ideoký hiệu bằng thuật toán Hidden Markov Model từ chuỗi ký tự khung xương được trích xuất ra.

Phương án 2: Thực hiện xây dựng mạng kết hợp Convoluntion Neural Network kết hợp với Long Short Term Memory để nhận dạng hành động từ video.

Phương án 3: Trích xuất hình dáng khung xương từ từng frame ảnh 2D bằng mạng CNN để xuất ra tọa độ khung xương trên hệ tọa độ 2D. Sử dụng một mạng Deep neural network để nhận dạng ngôn ngữ ký hiệu từ chuỗi khung xương đó.

Ở phương án 1 việc sử dụng thiết bị kinect sẽ là quá cồng kềnh để có thể mang theo, và sử dụng trong đời sống hàng ngày. Mô hình HMM có thể là một mô hình máy học cổ điển, còn nhiều nhược điểm hơn so với các mô hình mới hiện nay. Ở phương án 2, để phân loại hành động, cần có tập dữ liệu lớn vì với mỗi góc nhìn khác nhau sẽ cho ra một ảnh khác nhau và với cùng một hành động sẽ cho ra những đặc trưng khác nhau, như vậy sẽ khó để có thể xây dựng mô hình này. Phương án 3 được chọn vì có nhiều ưu điểm hơn phương án 1 và 2. Vì tính khả thi có thể ứng dụng được trên các thiết bị có camera mà người sử dụng có thể mang đi như điện thoại thông minh.

Ứng dụng hướng tới hỗ trợ cộng đồng người người khiếm thính trong giao tiếp thường ngày nên việc đầu tiên cần hướng đến là tính tiện lợi và dễ sử dụng. Vì vậy thiết bị giúp hỗ trợ người khiếm thính cần có thể dễ dàng mang đi gọn nhẹ. Do đó mục tiêu luận văn hướng đến là thực hiện phần mềm để có thể hoạt động trên điện thoại thông minh. 

\section{Nội dụng luận văn}
Nội dụng luận văn sẽ hướng đến xây dựng phần mềm nhận diện một số từ ngữ trong bộ ngôn ngữ ký hiệu của người khiếm thính ở TP.HCM. Các nội dụng này sẽ sắp xếp xoay quanh theo thứ tự hoạt động của mô hình nhận diện ngôn ngữ ký hiệu được xây dựng trong luận văn. Tổng quan, phần mềm nhận diện sẽ hoạt động theo các bước trong sơ đồ hình \ref{fig:diagram}.

\FloatBarrier
\begin{figure}[htp]
\begin{center}
\includegraphics[scale=0.7]{chap1/c1_figs/diagram.png}
\end{center}
\caption{Thứ tự hoạt động mô hình nhận dạng}
\label{fig:diagram}
\end{figure}
\FloatBarrier


\section{Bố cục trình bày}
\label{ss:bo_cuc_trinh_bay}

Toàn bộ nội dung của luận văn sẽ đi theo bố cục chặt chẽ theo thứ tự các bước mà luận văn đã nghiên cứu và thực hiện. Các nội dung chính của luận văn sẽ được chia thành các chương cụ thể để có thể dễ dàng xem xét, và nắm bắt vấn đề. Tại mỗi chương sẽ trình bày cụ thể từ lý thuyết của chương đó cho đến cụ thể các bước thực hiện cũng như kết luận từng chương.

Bố cục của luận văn sẽ được trình bày theo trình tự và những nội dung được khái quát trong bảng \ref{table:bo_cuc_luan_van}.

\FloatBarrier
\begin{table}[h]
\caption{Tổng quan các chương của luận văn}
\label{table:bo_cuc_luan_van}
\centering
\begin{center}
\begin{tabular}{|c|p{13cm}|} 
 \hline
Chương  & Nội dung \\
 \hline
 Chương 1 & Giới thiệu chung về ngôn ngữ ký hiệu; Các nghiên cứu về nhận dạng ngôn ngữ ký hiệu; sơ lược mục tiêu, tổng quan và cấu trúc các phần của luận văn.\\
 \hline 
 Chương 2 & Trình bày lý thuyết về học sâu, các thuật toán huấn luyện mạng và mạng mobile net.\\
 \hline 
 Chương 3 & Giới thiệu các nghiên cứu về ước lượng đặc trưng khung xương; Cách thức hoạt động của mạng ước tính đặc trưng khung xương mà luận văn sử dụng .\\
 \hline
 Chương 4 & Trình bày về mạng neural network được luận văn xây dựng để phân loại các từ trong ngôn ngữ ký hiệu; Cách xử lý dữ liệu đầu vào của mạng và cách huấn luyện mạng. \\
 \hline 
 Chương 5 & Giải thuật Deep Sort dùng để theo dõi từng người.\\
 \hline
 Chương 6 & Các thử nghiệm, kết quả và đánh g.iá\\
 \hline
 Chương 7 & Tổng kết.\\
 \hline
 
\end{tabular}
\end{center}
\end{table}
\FloatBarrier



\newpage
\chapter{CƠ SỞ LÝ THUYẾT}
\label{c:co_so_ly_thuyet}

%\section{CÁC LÝ THUYẾT LIÊN QUAN ĐẾN HỌC SÂU}
%\label{s:ly_thuyet_hoc_sau}

\section{MẠNG NEURAL NETWOK}
%Nguồn tham khao: https://dominhhai.github.io/vi/2018/04/nn-intro/

%				https://ujjwalkarn.me/2016/08/09/quick-intro-neural-networks/
				
%				https://towardsdatascience.com/machine-learning-for-beginners-an-introduction-to-neural-networks-d49f22d238f9
				
%				http://cs231n.github.io/neural-networks-1/
Con chó có thể phân biệt được người thân trong gia đình và người lạ hay đứa trẻ có thể phân biệt được các con vật. Những việc tưởng chừng như rất đơn giản nhưng lại cực kì khó để thực hiện bằng máy tính. Vậy sự khác biệt nằm ở đâu? Câu trả lời nằm ở bộ não với lượng lớn các neuron thần kinh liên kết với nhau. Vậy thì liệu máy tính có thể mô phỏng lại mô hình ấy để giải các bài toán trên ?

Neural là tính từ của neuron (nơ-ron), network chỉ cấu trúc đồ thị nên neural network (NN) là một hệ thống tính toán lấy cảm hứng từ sự hoạt động của các nơ-ron trong hệ thần kinh.

Mạng noron nhân tạo (Neural Network - NN) là một mô hình tính toán được lấy cảm hứng từ cách mạng noron sinh học trong não người xử lý thông tin. Kết hợp với các kĩ thuật học sâu (Deep Learning - DL), NN  trở thành một công cụ hiệu quả và có nhiều kết quả đột phá cho nhiều bài toán khó như nhận dạng ảnh, nhận dạng giọng nói thị giác máy tính và xử lý ngôn ngữ tự nhiên. Trong nội dung này, luận văn sẽ trình bày các lý thuyết cơ bản của mạng NN từ các thành phần cơ bản, kiến trúc mạng và các kỹ thuật huấn luyện (training) một mạng NN.

\subsection{Hoạt động của các neuron sinh học}

\FloatBarrier
\begin{figure}[htp]
\begin{center}
\includegraphics[scale=0.8]{chap2/c2_figs/neuron.png}
\end{center}
\caption{Cấu trúc một tế bào thần kinh}
\label{fig:neuronsinhhoc}
\end{figure}
\FloatBarrier
\centerline{Nguồn: https://askabiologist.asu.edu/neuron-anatomy}

Neuron là đơn vị cơ bản cấu tạo hệ thống thần kinh và là một phần quan trọng nhất của não. Não chúng ta gồm khoảng 10 triệu neuron và mỗi neuron liên kết với khoangr 10.000 neuron khác.

Ở mỗi neuron có phần thân (soma) chứa nhân, các tín hiệu đầu vào qua sợi nhánh (dendrites) và các tín hiệu đầu ra qua sợi trục (axon) kết nối với các neuron khác. Hiểu đơn giản mỗi neuron nhận dữ liệu đầu vào qua sợi nhánh và truyền dữ liệu đầu ra qua sợi trục, đến các sợi nhánh của các neuron khác.

Mỗi neuron nhận xung điện từ các neuron khác qua sợi nhánh. Nếu các xung điện này đủ lớn để kích hoạt neuron, thì tín hiệu này đi qua sợi trục đến các sợi nhánh của các neuron khác. => Ở mỗi neuron cần quyết định có kích hoạt neuron đấy hay không.

Tuy nhiên NN chỉ là lấy cảm hứng từ não bộ và cách nó hoạt động, chứ không phải bắt chước toàn bộ các chức năng của nó. Việc chính của chúng ta là dùng mô hình đấy đi giải quyết các bài toán chúng ta cần.

\subsection{Perceptron}
\label{s:perceptron}
\FloatBarrier
\begin{figure}[htp]
\begin{center}
\includegraphics[scale=0.6]{chap2/c2_figs/perceptron.PNG}
\end{center}
\caption{Perceptron}
\label{fig:perceptron}
\end{figure}
\FloatBarrier

Lấy ý tưởng từ neuron sinh học, neuron nhân tạo với tên gọi perceptron cũng họat động theo cách gần giống với neuron sinh học để tạo thành một mạng thần kinh nhân tạo cho máy tính.

Đơn vị tính toán cơ bản trong một mạng NN được gọi là perceptron và thường được gọi là \textbf{node} hay \textbf{unit}. Một node nhận các đầu vào từ các nodes khác hoặc từ nguồn bên ngoài, sau đó tính toán tạo ra giá trị ngõ ra. Mỗi ngõ vào có một trọng số liên kết và giá trị này biểu thi mức độ liên quan giữa node hiện tại và node trước nó. Node áp dụng một hàm $f$ (được giới thiệu ở nội dung bên dưới) vào tổng các các tích ngõ vào và trọng số để tạo giá trị ngõ ra. Hình \ref{fig:single_neuron} miêu tả chi tiết một neuron và các hoạt động của nó.

\FloatBarrier
\begin{figure}[htp]
\begin{center}
\includegraphics[scale=1]{chap2/c2_figs/single_neuron.PNG}
\end{center}
\caption{Một neuron cơ bản}
\label{fig:single_neuron}
\end{figure}
\FloatBarrier

Node trong hình \ref{fig:single_neuron} có hai đầu vào \textbf{Input 1} và \textbf{Input 2} có giá trị tương ứng $X1$ và $X2$, trọng số là \textbf{w1} và \textbf{w2}. Ngoài ra, có một đầu vào khác với giá trị \textbf{1} và trọng số b (được gọi là bias). Ngõ ra của node là Y được tính toán như hình \ref{fig:single_neuron}. Hàm \textbf{f} là hàm phi tuyến tính và được gọi là hàm kích hoạt (Activation Function). Đặc tính phi tuyến của hàm kích hoạt giúp mạng NN có thể "học" những dữ liệu thực trọng tự nhiên và hầu hết chúng đều có tính chất phi tuyến.

Các hàm kích hoạt thường được sử dụng trong thực tế:
\begin{itemize}
\item \textbf{Sigmoid:} lấy giá trị ngõ vào thực và ép nó nằm trong giới hạn [0, 1].
\begin{equation}
\sigma (x) = \frac{1}{{1 + {e^{ - x}}}}
\end{equation}
\item \textbf{Tanh:} lấy giá trị ngõ vào thực và ép nó nằm trong giới hạn [-1, 1].
\begin{equation}
\tanh (x) = 2\sigma (2x) - 1
\end{equation}
\item \textbf{ReLU:} lấy giá trị ngõ vào thực và lấy ngưỡng ở 0 (thay thế các giá trị âm bằng 0 hoặc giá trị rất nhỏ).
\begin{equation}
f(x) = \max (0,x)
\end{equation}
\end{itemize}

\noindent Đồ thị các hàm kích hoạt được mô tả trong hình \ref{fig:activation_function}.

\FloatBarrier
\begin{figure}[htp]
\begin{center}
\includegraphics[scale=0.8]{chap2/c2_figs/activation_function.PNG}
\end{center}
\caption{Đồ thị các hàm kích hoạt}
\label{fig:activation_function}
\end{figure}
\FloatBarrier

Tầm quan trọng của bias: 

\subsection{Kiến trúc mạng Neuron Network}

Một mạng NN được xây dựng gồm nhiều lớp (layer). Mỗi lớp được cấu thành từ nhiều node cơ bản (đã trình bày trong mục \ref{s:perceptron}). Ngõ ra của các node ở lớp phía trước là ngõ vào của các node lớp phía sau, chúng được gọi là các liên kết (connection) và tương ứng với các trọng số khác nhau.

\FloatBarrier
\begin{figure}[htp]
\begin{center}
\includegraphics[scale=0.8]{chap2/c2_figs/structure_NN.PNG}
\end{center}
\caption{Mạng Neural Network cơ bản}
\label{fig:structure_NN}
\end{figure}
\FloatBarrier

Một mạng NN cơ bản sẽ có 3 tầng (minh họa trong hình \ref{fig:structure_NN}):
\begin{itemize}
\item \textbf{Tầng vào} (\textit{input layer}): Là tầng bên trái cùng của mạng thể hiện cho các đầu vào của mạng.
\item \textbf{Tầng ra} (\textit{output layer}): Là tầng bên phải cùng của mạng thể hiện cho các đầu ra của mạng.
\item \textbf{Tầng ẩn} (\textit{hidden layer}): Là tầng nằm giữa tầng vào và tầng ra thể hiện cho việc suy luận logic của mạng.
\end{itemize}

Một mạng NN chỉ có một tầng vào và một tầng ra, nhưng có thể có nhiều tầng ẩn. Số lượng tầng ẩn phụ thuộc vào độ phức tạp của bài toán được mạng NN giải quyết và được thiết kế dựa trên kinh nghiệm của người xây dựng mạng.

\subsection{Hoạt động của mạng}
\label{hoat_dong_cua_mang}

Tín hiệu đầu vào (gồm các thông tin cần dự đoán) sẽ được truyền từ input layer. Sau đó được tính toán qua các hidden layer bới các nodes. Cuối cùng output layer sẽ thực hiện việc dự đoán và phân lọai.\\

Mỗi node trong hidden layer và output layer sẽ thực hiện các công việc sau:
\begin{itemize}
\item Liên kết với tất cả các node ở layer trước đó với các hệ số $w$ riêng.
\item Mỗi node có 1 hệ số bias b riêng.
\item Diễn ra 2 bước: tính tổng linear và áp dụng activation function đưa ra output của node.
\end{itemize}

Để hiểu rõ ràng nhất, ta đi sâu vào các tính toán trong một mạng NN cụ thể như hình \ref{fig:NN}.\\

\FloatBarrier
\begin{figure}[htp]
\begin{center}
\includegraphics[scale=1]{chap2/c2_figs/nn_full-2.png}
\end{center}
\caption{Mô hình neural network trên gồm 3 layer. Input layer có 2 node $(l^{(0)} = 2$, hidden layer 1 có 3 node, hidden layer 2 có 3 node và output layer có 1 node.}
\label{fig:NN}
\end{figure}
\FloatBarrier

\textbf{Ký hiệu:}
\begin{itemize}
\item Số node trong hidden layer thứ $i$ là $l^{(i)}l(i)$.

\item Ma trận $W^{(k)}$ kích thước $l^{(k-1)}*l^{(k)}$ là ma trận hệ số giữa layer $(k-1)$ và layer $k$, trong đó $w_{ij}^{(k)}$ là hệ số kết nối từ node thứ $i$ của layer $k-1$ đến node thứ $j$ của layer $k$.

\item Vector $b^{(k)}$ kích thước $l^{k} * 1$ là hệ số bias của các node trong layer $k$, trong đó $b_i^{(k)}$ là bias của node thứ $i$ trong layer $k$. 
\end{itemize}

Với node thứ $i$ trong layer $l$ có bias $b_i^{(l)}$ thực hiện 2 bước:
\begin{itemize}
\item Tính tổng linear: $z_i^{(l)} = \sum_{j=1}^{l^{(l-1)}} a_j^{(l-1)} * w_{ji}^{(l)} + b_i^{(l)}$ là tổng tất cả các node trong layer trước nhân với hệ số w tương ứng, rồi cộng với bias b.
\item Áp dụng activation function: $a_i^{(l)} = \sigma(z_i^{(l)})$
\end{itemize}

Vector $z^{(k)}$ kích thước $l^{(k)} * 1$ là giá trị các node trong layer $k$ sau bước tính tổng linear.

Vector $a^{(k)}$ kích thước $l^{(k)} * 1$ là giá trị của các node trong layer $k$ sau khi áp dụng hàm activation function.



Do mỗi node trong hidden layer và output layer đều có bias nên trong input layer và hidden layer cần thêm node 1 để tính bias (nhưng không tính vào tổng số node layer có).

Tại node thứ 2 ở layer 1, ta có:

\begin{itemize}
\item $z_2^{(1)} =  x_1 * w_{12}^{(1)} +  x_2 * w_{22}^{(1)} + b_2^{(1)}$
\item $a_2^{(1)} = \sigma(z_2^{(1)})$
\end{itemize} 

Hay ở node thứ 3 layer 2, ta có:
\begin{itemize}
\item $z_3^{(2)} =  a_1^{(1)} * w_{13}^{(2)} + a_2^{(1)} * w_{23}^{(2)}  + a_3^{(1)} * w_{33}^{(2)} + b_3^{(2)}$
\item $a_2^{(1)} = \sigma(z_2^{(1)})$
\end{itemize} 

\subsection{Quá trình huấn luyện một mạng NN}

Quá trình huấn luyện một mạng NN được thể hiện qua sự lặp đi lặp lại hai bước sau:
\begin{itemize}
\item \textbf{Feedforward:} Lan truyền tiến. Dự đoán output $\hat{y}$ với một input $x$ bằng cách tính toán từ đầu đến cuối của mạng neuron.
\item \textbf{Backpropagation:} Lan truyền ngược và cập nhật trọng số.
\end{itemize}
\textbf{Bước 1: Lan truyền tiến}
Để nhất quán về mặt ký hiệu, gọi input layer là $a^{(0)} (=x)$ kích thước $2*1$.


\FloatBarrier
\begin{figure}[htp]
\begin{center}
\includegraphics[scale=0.8]{chap2/c2_figs/feed_forward.jpg}
\end{center}
\label{fig:feed_forward0}
\end{figure}
\FloatBarrier
Tương tự ta có:
$\newline z^{(2)} = (W^{(2)})^T * a^{(1)} + b^{(2)}\newline  a^{(2)} = \sigma(z^{(2)}) \newline z^{(3)} = (W^{(3)})^T * a^{(2)} + b^{(3)}\newline  \hat{y} = a^{(3)} = \sigma(z^{(3)})$

\FloatBarrier
\begin{figure}[htp]
\begin{center}
\includegraphics[scale=0.75]{chap2/c2_figs/ff.png}
\end{center}
\caption{Feedforward}
\label{fig:feed_forward}
\end{figure}
\FloatBarrier

\begin{itemize}
\item[$\square$] \textbf{Biểu diễn dưới dạng ma trận:}
\end{itemize}
Tuy nhiên khi làm việc với dữ liệu ta cần tính dự đoán cho nhiều dữ liệu một lúc, nên gọi $X$ là ma trận $n*d$, trong đó $n$ là số dữ liệu và $d$ là số trường trong mỗi dữ liệu, trong đó $x_j^{[i]}$ là giá trị trường dữ liệu thứ $j$ của dữ liệu thứ $i$.
Biểu diễn dạng ma trận của vector dữ liệu đầu vào như sau:

\FloatBarrier
\begin{figure}[htp]
\begin{center}
\includegraphics[scale=0.8]{chap2/c2_figs/1.jpg}
\end{center}
\label{fig:feed_forward}
\end{figure}
\FloatBarrier

Do $x^{[1]}$ là vector kích thước $d*1$ tuy nhiên ở $X$ mỗi dữ liệu được viết theo hàng nên cần transpose $x^{[1]}$ thành kích thước $1*d$, kí hiệu: $(x^{[1]})^T$
Gọi ma trận $Z^{(i)}$ kích thước$ N*l^{(i)}$ trong đó $z_{j}^{(i)[k]}$ là giá trị thứ $j$ trong layer $i$ sau bước tính tổng linear của dữ liệu thứ $k$ trong dataset.

*** Kí hiệu $(i)$ là layer thứ $i$ và kí hiệu $[k]$ là dữ liệu thứ $k$ trong dataset.

Tương tự, gọi ma trận $A^{(i)}$ kích thước $N*l^{(i)}$ trong đó $a_{j}^{(i)[k]}$ là giá trị thứ $j$ trong layer $i$ sau khi áp dụng activation function của dữ liệu thứ $k$ trong dataset.

\FloatBarrier
\begin{figure}[htp]
\begin{center}
\includegraphics[scale=0.8]{chap2/c2_figs/2.jpg}
\end{center}
\label{fig:feed_forward}
\end{figure}
\FloatBarrier

Do đó:

\FloatBarrier
\begin{figure}[htp]
\begin{center}
\includegraphics[scale=0.75]{chap2/c2_figs/3.jpg}
\end{center}
\label{fig:feed_forward}
\end{figure}
\FloatBarrier
Như vậy:

\FloatBarrier
\begin{figure}[htp]
\begin{center}
\includegraphics[scale=0.75]{chap2/c2_figs/4.jpg}
\end{center}
\label{fig:feed_forward}
\end{figure}
\FloatBarrier

Vậy là có thể tính được giá trị dự đoán của nhiều dữ liệu 1 lúc dưới dạng ma trận.

Giờ từ input $X$ ta có thể tính được giá trị dự đoán $\hat{Y}$, tuy nhiên việc chính cần làm là đi tìm hệ số $W$ và $b$. Có thể nghĩ ngay tới thuật toán gradient descent và việc quan trọng nhất trong thuật toán gradient descent là đi tìm đạo hàm của các hệ số đối với loss function. Và việc tính đạo hàm của các hệ số trong neural network được thực hiện bởi thuật toán backpropagation, sẽ được trình bày ở bước sau.

\textbf{Bước 2: Backpropagation - Lan truyền ngược và cập nhật trọng số}
Giờ ta cần đi tìm hệ số $W$ và $b$. Có thể nghĩ ngay tới thuật toán gradient descent và việc quan trọng nhất trong thuật toán gradient descent là đi tìm đạo hàm của các hệ số đối với loss function. Bước này sẽ tính đạo hàm của các hệ số trong neural network với thuật toán backpropagation.

Quá trình học vẫn là tìm lấy một hàm lỗi để đánh giá và tìm cách tối ưu hàm lỗi đó để được kết quả hợp lý nhất có thể. Với mỗi điểm $(x^{[i]}, y_i)$ ta có hàm loss function được tính theo công thức: $$L = -(y_i * log(\hat{y_i}) + (1 - y_i) * log(1 - \hat{y_i}))$$

Hàm loss function trên toàn bộ dữ liệu:
$$J = - \sum_{i=1}^{N}(y_i * log(\hat{y_i}) + (1 - y_i) * log(1 - \hat{y_i}))$$

\begin{itemize}
\item[$\blacksquare$] \textbf{Gradient Descent}
\end{itemize}
Để áp dụng gradient descent ta cần tính được đạo hàm của các hệ số W và bias b với hàm loss function.
*** Kí hiệu chuẩn về đạo hàm
\begin{itemize}
\item Khi hàm f(x) là hàm 1 biến x, ví dụ: $ f(x) = 2*x + 1$. Đạo hàm của $f$ đối với biến $x$ kí hiệu là $\frac{df}{dx}\newline $
\item Khi hàm $f(x, y)$ là hàm nhiều biến, ví dụ $ f(x, y) = x^2 + y^2$. Đạo hàm $f$ với biến $x$ kí hiệu là $ \frac{\partial f}{\partial x}$
\end{itemize}

Với mỗi điểm $(x^{([i]}, y_i)$, hàm loss function sẽ là:

$$L = -(y_i * log(\hat{y_i}) + (1 - y_i) * log(1 - \hat{y_i}))$$


trong đó: $\hat{y_i} = a_1^{(2)} = \sigma(a_1^{(1)} * w_{11}^{(2)} + a_2^{(1)} * w_{21}^{(2)} + b_1^{(2)})$
 là giá trị mà model dự đoán, còn $y_i$ là giá trị thật của dữ liệu.
$$\frac{\partial L}{\partial\hat{y_i}} = - \frac{\partial(y_i * log(\hat{y_i}) + (1 - y_i) * log(1 - \hat{y_i}))}{\partial\hat{y_i}}= - (\frac{y_i}{\hat{y_i}} - \frac{1-y_i}{(1-\hat{y})})\newline$$
Tính đạo hàm L với $W^{(2)}$, $b^{(2)}$\\
Áp dụng chain rule ta có: $$ \frac{\partial L}{\partial b_1^{(2)}} = \frac{dL}{d\hat{y_i}} * \frac{\partial\hat{y_i}}{\partial b_1^{(2)} } $$

\FloatBarrier
\begin{figure}[htp]
\begin{center}
\includegraphics[scale=1]{chap2/c2_figs/6.png}
\end{center}
\label{fig:feed_forward}
\end{figure}
\FloatBarrier

Từ đồ thị ta thấy:

$$\frac{\partial\hat{y_i}}{\partial b_1^{(2)}} = \hat{y_i} * (1-\hat{y_i})\newline$$

$$\frac{\partial\hat{y_i}}{\partial w_{11}^{(2)}} = a_1^{(1)}*\hat{y_i} * (1-\hat{y_i})\newline$$

$$\frac{\partial\hat{y_i}}{\partial w_{21}^{(2)}} = a_2^{(1)}*\hat{y_i} * (1-\hat{y_i})\newline$$

$$\frac{\partial\hat{y_i}}{\partial a_1^{(1)} }=w_{11} ^{(2)}*\hat{y_i} * (1-\hat{y_i})\newline$$

$$\frac{\partial\hat{y_i}}{\partial a_2^{(1)} }=w_{21} ^{(2)}*\hat{y_i} * (1-\hat{y_i})\newline$$
Do đó:
$$\frac{\partial L}{\partial b_1^{(2)}} = \frac{\partial L}{\partial\hat{y_i}} * \frac {\partial\hat{y_i}}{\partial b_1^{(2)}} = - (\frac{y_i}{\hat{y_i}} - \frac{1-y_i}{(1-\hat{y_i})}) * \hat{y_i} * (1-\hat{y_i}) = -(y_i * (1-\hat{y_i}) - (1-y_i) * \hat{y_i})) = \hat{y_i}-y_i$$

Tương tự:
$$\frac{\partial L}{\partial w_{11} ^ {(2)}} = a_1 ^ {(1)} *     (\hat{y_i}-y_i)\newline \newline$$

$$\frac{\partial L}{\partial w_{21} ^ {(2)}} = a_2 ^ {(1)} *     (\hat{y_i}-y_i) \newline \newline$$

$$\frac{\partial L}{\partial a_1 ^ {(1)}} =  w_{11} ^ {(2)}  *     (\hat{y_i}-y_i) \newline \newline$$

$$\frac{\partial L}{\partial a_2 ^ {(1)}} =  w_{21} ^ {(2)}  *     (\hat{y_i}-y_i) \newline \newline $$


\begin{itemize}
\item[$\square$] \textbf{Biểu diễn dưới dạng ma trận:}
\end{itemize} 
\textbf{*** Lưu ý:} đạo hàm của L đối với ma trận W kích thước m*n cũng là một ma trận cùng kích thước $m*n$.

\FloatBarrier
\begin{figure}[htp]
\begin{center}
\includegraphics[scale=1]{chap2/c2_figs/5.jpg}
\end{center}
\label{fig:feed_forward}
\end{figure}
\FloatBarrier

Do đó:
$$\frac{\partial J}{\partial W^{(2)}} = (A^{(1)})^T * (\hat{Y} - Y), \frac{\partial J}{\partial b^{(2)}} = (sum(\hat{Y} - Y))^T,  \frac{\partial J}{\partial A^{(1)}} = (\hat{Y} - Y) * (W^{(2)})^T$$ 
là phép tính sum tính tổng các cột của ma trận.

\FloatBarrier
\begin{figure}[htp]
\begin{center}
\includegraphics[scale=0.8]{chap2/c2_figs/6.jpg}
\end{center}
\label{fig:feed_forward}
\end{figure}
\FloatBarrier

Vậy là đã tính xong đạo hàm của $L$ với hệ số $W^{(2)}, b^{(2)}$. Giờ sẽ đi tính đạo hàm của $L$ với hệ số $W^{(1)}, b^{(1)}$ để khi tính đạo hàm của hệ số và bias trong layer trước đấy sẽ cần dùng đến.

Tính đạo hàm L với $W^{(1)}, b^{(1)}$
Do $ a_1^{(1)} = \sigma(b_1^{(1)} + x_1*w_{11}^{(1)} + x_2*w_{21}^{(1)})$ 

Áp dụng chain rule ta có: $$ \frac{\partial L}{\partial b_1^{(1)}} = \frac{\partial L}{\partial a_1^{(1)}} * \frac{\partial a_1^{(1)}}{\partial b_1^{(1)} }$$
Ta có:
$$\frac{\partial a_1^{(1)}}{\partial b_1^{(1)}} = \frac{\partial a_1^{(1)}}{z_1^{(1)}} * \frac{z_1^{(1)}}{\partial b_1^{(1)}} = a_1^{(1)} * (1 - a_1^{(1)})$$
Do đó:
$$\frac{\partial L}{\partial b_1^{(1)}} = a_1 ^ {(1)} * (1 - a_1^{(1)}) * w_{11}^{(2)} * (\hat{y_i} - y_i)$$

Tương tự:
$$\frac{\partial L}{\partial w_{11}^{(1)}} = x_1 * a_1 ^ {(1)} * (1 - a_1^{(1)}) * w_{11}^{(2)} *  (\hat{y_i} - y_i)  \newline$$

$$\frac{\partial L}{\partial w_{12}^{(1)}} = x_1 * a_2 ^ {(1)} * (1 - a_2^{(1)}) * w_{11}^{(2)} *  (\hat{y_i} - y_i)  \newline$$

$$\frac{\partial L}{\partial w_{21}^{(1)}} = x_2 * a_1 ^ {(1)} * (1 - a_1^{(1)}) * w_{21}^{(2)} *  (\hat{y_i} - y_i)  \newline $$

$$\frac{\partial L}{\partial w_{22}^{(1)}} = x_2 * a_2^ {(1)} * (1 - a_2^{(1)}) * w_{21}^{(2)} *  (\hat{y_i} - y_i)  \newline$$

Có thể tạm viết dưới dạng chain rule là: $$\frac{\partial J}{\partial W^{(1)}} = \frac{\partial J}{\partial A^{(1)}} * \frac{\partial A^{(1)}}{\partial Z^{(1)}}* \frac{\partial Z^{(1)}}{\partial W^{(1)}} (1) $$

Từ trên đã tính được: $$\frac{\partial J}{\partial A^{(1)}} = (\hat{Y} - Y) * (W^{(2)})^T$$

Đạo hàm của hàm sigmoid: $\frac{d\sigma(x)}{dx} = \sigma(x) * (1 - \sigma(x))$ và $A^{(1)} = \sigma(Z^{(1)})$ , nên trong (1) có thể hiểu là $\frac{\partial A^{(1)}}{\partial Z^{(1)}} = A^{(1)}* (1 - A^{(1)})$

Cuối cùng, $Z^{(1)} = X * W^{(1)} + b^{(1)}$ nên có thể tạm hiểu $\frac{\partial Z^{(1)}}{\partial W^{(1)}} = X$ , nó giống như $f(x)= a*x +b$ $ => \frac{df}{dx} = a$ .

Kết hợp tất cả lại ta được:
$$\frac{\partial J}{\partial W^{(1)}} = X^T * (((\hat{Y} – Y) * (W^{(2)})^T)\otimes A^{(1)}\otimes (1-A^{(1)}) ) $$

Vậy khi nào cần dùng element-wise $(\otimes)$, khi nào dùng nhân ma trận $(*)$?
\begin{itemize}
\item Khi tính đạo hàm ngược lại qua bước activation thì dùng $(\otimes)$.
\item Khi có phép tính nhân ma trận thì dùng $(*)$, nhưng đặc biệt chú ý đến \textbf{kích thước ma trận} và dùng \textbf{transpose} nếu cần thiết. Ví dụ: ma trận $X$ kích thước $N*3$, W kích thước $3*4$, $Z = X * W$ sẽ có kích thước $N*4$ thì $\frac{\partial J}{\partial W} = X^T * (\frac{\partial J}{\partial Z})$ và $\frac{\partial J}{\partial X} = (\frac{\partial J}{\partial Z}) * W^T$.
\end{itemize}
Tương tự: $$\frac{\partial L}{\partial b^{(1)}} = sum(((\hat{Y} – Y) * (W^{(2)})^T)\otimes A^{(1)})^T$$

Vậy là đã tính xong hết đạo hàm của loss function với các hệ số $W$ và bias $b$, giờ có thể áp dụng gradient descent để giải bài toán.

Giờ thử tính $ \frac{\partial L}{\partial x_1}$, ở bài này thì không cần vì chỉ có 1 hidden layer, nhưng nếu nhiều hơn 1 hidden layer thì cần phải tính bước này để tính đạo hàm với các hệ số trước đó.

\FloatBarrier
\begin{figure}[htp]
\begin{center}
\includegraphics[scale=0.75]{chap2/c2_figs/1.png}
\end{center}
\caption{Đường màu đỏ cho $w_{11}^{(1)}$, đường màu xanh cho $x_1$}
\label{fig:feed_forward}
\end{figure}
\FloatBarrier

Ta thấy $w_{11}^{(1)}$ chỉ tác động đến $a_1^{(1)}$, cụ thể là $ a_1^{(1)} = \sigma(b_1^{(1)} + x_1*w_{11}^{(1)} + x_2*w_{21}^{(1)})$ 

Tuy nhiên $x_1$ không những tác động đến $a_1^{(1)}$ mà còn tác động đến $a_2^{(1)}$, nên khi áp dụng chain rule tính đạo hàm của $L$ với $x_1$ cần tính tổng đạo hàm qua cả $a_1^{(1)}$ và $a_2^{(1)}$.

\FloatBarrier
\begin{figure}[htp]
\begin{center}
\includegraphics[scale=0.75]{chap2/c2_figs/2.png}
\end{center}
\caption{backpropagation tác động trong lớp ẩn}
\label{fig:feed_forward}
\end{figure}
\FloatBarrier

Do đó:
$$\frac{\partial L}{\partial x_1} = \frac{\partial L}{\partial a_1^{(1)}} * \frac{\partial a_1^{(1)}}{\partial x_1} + \frac{\partial L}{\partial a_2^{(1)}} * \frac{\partial a_2^{(1)}}{\partial x_1} =  w_{11}^{(1)}* a_1 ^ {(1)} * (1 – a_1^{(1)}) * w_{11}^{(2)} * (y_i – \hat{y_i}) + w_{12}^{(1)}* a_2 ^ {(1)} * (1 – a_2^{(1)}) * w_{21}^{(2)} * (y_i – \hat{y_i})  \newline$$

Sau tất cả, mô hình tổng quát sẽ bao gồm các bước như sau:

\FloatBarrier
\begin{figure}[htp]
\begin{center}
\includegraphics[scale=0.75]{chap2/c2_figs/3.png}
\end{center}
\caption{Mô hình neural network}
\label{fig:feed_forward}
\end{figure}
\FloatBarrier

\begin{itemize}
\item \textbf{Bước 1:} Tính $\frac{\partial J}{\partial \hat{Y}}$, trong đó $\hat{Y} = A^{(3)}$
\item \textbf{Bước 2:} Tính $$\frac{\partial J}{\partial \hat{W^{(3)}}}= (A^{(2)})^T * (\frac{\partial J}{\partial \hat{Y}} \otimes \frac{\partial A^{(3)}}{\partial Z^{(3)}}),  \frac{\partial J}{\partial \hat{b^{(3)}}}= (sum( \frac{\partial J}{\partial \hat{Y}} \otimes \frac{\partial A^{(3)}}{\partial Z^{(3)}}))^T$$
\item \textbf{Bước 3:} Tính $$\frac{\partial J}{\partial \hat{W^{(2)}}}= (A^{(1)})^T * (\frac{\partial J}{\partial A^{(2)}} \otimes \frac{\partial A^{(2)}}{\partial Z^{(2)}}),  \frac{\partial J}{\partial \hat{b^{(2)}}}= (sum (\frac{\partial J}{\partial A^{(2)}} \otimes \frac{\partial A^{(2)}}{\partial Z^{(2)}}))^T$$ và tính $$\frac{\partial J}{\partial \hat{A^{(1)}}}= ( \frac {\partial J}{\partial A^{(2)}} \otimes \frac{\partial A^{(2)}}{\partial Z^{(2)}}) * (W^{(2)})^T$$ 
\item \textbf{Bước 4:} Tính $$\frac{\partial J}{\partial \hat{W^{(1)}}}= (A^{(0)})^T  * (\frac{\partial J}{\partial A^{(1)}} \otimes \frac{\partial A^{(1)}}{\partial Z^{(1)}}),  \frac{\partial J}{\partial \hat{b^{(1)}}}= (sum (\frac{\partial J}{\partial A^{(1)}} \otimes \frac{\partial A^{(1)}}{\partial Z^{(1)}}))^T$$ , trong đó $A^{(0)} = X$
\end{itemize}

Nếu network có nhiều layer hơn thì cứ tiếp tục cho đến khi tính được đạo hàm của loss function $J$ với tất cả các hệ số $W$ và bias $b$.

Nếu hàm activation là sigmoid thì $\frac{\partial A^{(i)}}{\partial Z^{(i)}} = A^{(i)} \otimes (1-A^{(i)})$

Tổng kết lại, 2 quá trình Feedfoward và Backpropagation sẽ diễn ra lần lượt như sau:

\FloatBarrier
\begin{figure}[htp]
\begin{center}
\includegraphics[scale=0.7]{chap2/c2_figs/7.png}
\end{center}
\end{figure}
\FloatBarrier

\FloatBarrier
\begin{figure}[htp]
\begin{center}
\includegraphics[scale=0.65]{chap2/c2_figs/8.png}
\end{center}
\caption{Feedforward và Backpropagation}
\label{fig:backpropagation}
\end{figure}
\FloatBarrier

(Nội dung dựa theo "Sách Deep Learning cơ bản" - tác giả: Nguyễn Thanh Tuấn)

\section{Mạng Convolutional Neural Network (CNN)}
Convolutional Neural Network (CNNs – Mạng nơ-ron tích chập) là một trong những mô hình Deep Learning tiên tiến giúp cho chúng ta xây dựng được những hệ thống thông minh với độ chính xác cao như hiện nay. Trong luận văn này, sẽ trình bày về Convolution (tích chập) đi từ những khái niệm cơ bản nhất đến ứng dụng của nó cũng như ý tưởng của mô hình CNNs trong phát hiện và trích xuất đặc trưng khung xương từ ảnh RGB.

Mạng Neural Network truyền thống tuy đã giải quyết  được một số vấn đề lớn lúc bấy giờ nhưng lại gặp một số khó khăn khi giải quyết bài toán xử lý, phân loại hình ảnh.
Đối với mạng Neural Network truyền thống khi xử lý ảnh màu 64*64 được biểu diễn dưới dạng 1 tensor 64*64*3. Việc để biểu thị hết nội dung của bức ảnh thì cần truyền vào input layer tất cả các pixel (64*64*3 = 12288). Nghĩa là input layer giờ có 12288 nodes. Giả sử số lượng node trong hidden layer 1 là 1000. Số lượng weight $W$ giữa input layer và hidden layer 1 là 12288*1000 = 12288000, số lượng bias là 1000 $=>$ tổng số parameter là: 12289000. Đây mới chỉ là số parameter giữa input layer và hidden layer 1, trong model còn nhiều layer nữa, và nếu kích thướcây ảnh tăng, ví dụ 512*512 thì số lượng parameter tăng cực kì nhanh. Điều này khiến cho việc tính toán của máy tính cần rất nhiều công sức nhưng lại không mang lại hiệu quả cao. Do vậy ta cần có giải pháp tốt hơn.

Nhận xét:
\begin{itemize}
\item Trong ảnh các pixel ở cạnh nhau thường có liên kết với nhau hơn là những pixel ở xa. Ví dụ để thể hiện một vật thể trên ảnh cần các pixel gần nhau và có màu sắc tương tự nhau.
\item Ngoài ra để so sánh các đối tượng là giống hay khác nhau cần phải so sánh giữa khu vực này với khu vực kia của bức ảnh. Do vậy cần phải có một bộ hệ số tính toán với các pixel quét hết toàn bộ bức ảnh để so sánh các vùng. Hay nói cách khác là các pixel ảnh chia sẻ hệ số với nhau.
\end{itemize}
=> Do vậy ý tưởng sử dụng mạng Convolutional Neural Network ra đời. Áp dụng phép tính convolution vào layer trong neural network ta có thể giải quyết được vấn đề lượng lớn parameter mà vẫn lấy ra được các đặc trưng của ảnh.
\subsection{Phép Tính Convolution}
\label{ss: convolution}
Để cho dễ hình dung mình sẽ lấy ví dụ trên ảnh xám, tức là ảnh được biểu diễn dưới dạng ma trận $A$ kích thước $m*n$.
Ta định nghĩa kernel là một ma trận vuông kích thước $k*k$ trong đó $k$ là số lẻ. $k$ có thể bằng $1, 3, 5, 7, 9,…$ Ví dụ kernel kích thước $3*3$.

Kí hiệu phép tính convolution $(\otimes)$, kí hiệu $Y = X \otimes W$.

Với mỗi phần tử $x_{ij}$ trong ma trận $X$ lấy ra một ma trận có kích thước bằng kích thước của kernel $W$ có phần tử $x_{ij}$ làm trung tâm (đây là vì sao kích thước của kernel thường lẻ) gọi là ma trận $A$. Sau đó tính tổng các phần tử của phép tính element-wise của ma trận $A$ và ma trận $W$, rồi viết vào ma trận kết quả $Y$.

\FloatBarrier
\begin{figure}[htp]
\begin{center}
\includegraphics[scale=0.65]{chap2/c2_figs/9.png}
\end{center}
\caption{Phép tính Convolution}
\label{fig:convolution}
\end{figure}
\FloatBarrier

Ví dụ khi tính tại $x_{22}$ (ô khoanh đỏ trong hình \ref{fig:convolution}), ma trận $A$ cùng kích thước với $W$, có $x_{22}$ làm trung tâm có màu nền da cam như trong hình. Sau đó tính $y_{11} = sum(A \otimes W) = x_{11}*w_{11} + x_{12}*w_{12} + x_{13}*w_{13} + x_{21}*w_{21} + x_{22}*w_{22} + x_{23}*w_{23} + x_{31}*w_{31} + x_{32}*w_{32} + x_{33}*w_{33} = 4$. Và làm tương tự với các phần tử còn lại trong ma trận.

Vì tâm của kernel $W$ không thể lướt hết ma trận $X$ nên $Y$ sẽ có kích thước nhỏ hơn ma trận $X$. Kích thước của ma trận Y là (m-k+1) * (n-k+1).

\FloatBarrier
\begin{figure}[htp]
\begin{center}
\includegraphics[scale=0.65]{chap2/c2_figs/10.png}
\end{center}
\caption{Convolution feature có kích thước nhỏ hơn ảnh ban đầu}
\label{fig:convolution-feature}
\end{figure}
\FloatBarrier

\begin{itemize}
\item[$\blacksquare$] \textbf{Padding}
Mỗi lần thực hiện phép tính convolution xong thì kích thước ma trận Y đều nhỏ hơn X. Tuy nhiên giờ ta muốn ma trận Y thu được có kích thước bằng ma trận X vì vậy cần tìm cách giải quyết cho các phần tử ở viền bằng cách thêm giá trị 0 ở viền ngoài ma trận X.

\FloatBarrier
\begin{figure}[htp]
\begin{center}
\includegraphics[scale=0.65]{chap2/c2_figs/10.png}
\end{center}
\caption{Ma trận X có viền 0 bên ngoài}
\label{fig:padding}
\end{figure}
\FloatBarrier

Rõ ràng là giờ đã giải quyết được vấn đề tìm A cho phần tử $x_{11}$, và ma trận $Y$ thu được sẽ bằng kích thước ma trận $X$ ban đầu.

Phép tính này gọi là convolution với $padding=1$. $Padding=k$ nghĩa là thêm $k$ vector $0$ vào mỗi phía của ma trận.

\item[$\blacksquare$] \textbf{Stride}
Như ở trên ta thực hiện tuần tự các phần tử trong ma trận $X$, thu được ma trận $Y$ cùng kích thước ma trận $X$, ta gọi là $stride=1$.

\FloatBarrier
\begin{figure}[htp]
\begin{center}
\includegraphics[scale=0.65]{chap2/c2_figs/11.png}
\end{center}
\caption{$stride=1, padding=1$}
\label{fig:padding}
\end{figure}
\FloatBarrier

Tuy nhiên nếu $stride=k (k > 1)$ thì ta chỉ thực hiện phép tính convolution trên các phần tử $x_{1+i*k,1+j*k}$. Ví dụ $k = 2$.

\FloatBarrier
\begin{figure}[htp]
\begin{center}
\includegraphics[scale=0.65]{chap2/c2_figs/12.png}
\end{center}
\caption{$padding=1, stride=2$}
\label{fig:padding,stride}
\end{figure}
\FloatBarrier

Hiểu đơn giản là bắt đầu từ vị trí $x_{11}$ sau đó nhảy $k$ bước theo chiều dọc và ngang cho đến hết ma trận $X$.

Kích thước của ma trận $Y$ là $3*3$ đã giảm đi đáng kể so với ma trận $X$.
Công thức tổng quát cho phép tính convolution của ma trận $X$ kích thước $m*n$ với kernel kích thước $k*k$, $stride = s$, $padding = p$ ra ma trận $Y$ kích thước $$(\frac{m-k+2p}{s}+1) * (\frac{n-k+2p}{s}+1)$$
Stride thường dùng để giảm kích thước của ma trận sau phép tính convolution.
Ý nghĩa của phép tính convolution:
Mục đích của phép tính convolution trên ảnh là làm mở, làm nét ảnh; xác định các đường;… Mỗi kernel khác nhau thì sẽ phép tính convolution sẽ có ý nghĩa khác nhau. 
\end{itemize} 
\subsection{Phép convolution trong mạng Neuron Network}
Với ảnh màu có tới 3 channels red, green, blue nên khi biểu diễn ảnh sẽ dưới dạng tensor 3 chiều. Nên ta cũng sẽ định nghĩa kernel là 1 tensor 3 chiều kích thước $k*k*3$.

\FloatBarrier
\begin{figure}[htp]
\begin{center}
\includegraphics[scale=0.5]{chap2/c2_figs/13.png}
\end{center}
\caption{Phép tính convolution trên ảnh màu với k=3.}
\label{fig:padding,stride}
\end{figure}
\FloatBarrier

Ta định nghĩa kernel có cùng độ sâu (depth) với biểu diễn ảnh, rồi sau đó thực hiện di chuyển khối kernel tương tự như khi thực hiện trên ảnh xám.

\FloatBarrier
\begin{figure}[htp]
\begin{center}
\includegraphics[scale=1]{chap2/c2_figs/14.png}
\end{center}
\caption{Tensor $X$ và $W$ 3 chiều được viết dưới dạng 3 matrix.}
\label{fig:padding,stride}
\end{figure}
\FloatBarrier

Khi biểu diễn ma trận ta cần 2 chỉ số hàng và cột: $i$ và $j$, thì khi biểu diễn ở dạng tensor 3 chiều cần thêm chỉ số độ sâu $k$. Nên chỉ số mỗi phần tử trong tensor là $x_{ijk}$.
$$y_{11} = b + (x_{111}*w_{111} +  x_{121}*w_{121} + x_{131}*w_{131} +  x_{211}*w_{211} +  x_{221}*w_{221} +  x_{231}*w_{231} +  x_{311}*w_{311} + $$ $$ x_{321}*w_{321} +  x_{331}*w_{331}) + (x_{112}*w_{112} +  x_{122}*w_{122} + x_{132}*w_{132} +  x_{212}*w_{212} +  x_{222}*w_{222} + $$ $$x_{232}*w_{232} +  x_{312}*w_{312} +  x_{322}*w_{322} +  x_{332}*w_{332}) +  (x_{113}*w_{113} +  x_{123}*w_{123} + x_{133}*w_{133} +  x_{213}*w_{213} + $$ $$ x_{223}*w_{223} +  x_{233}*w_{233} +  x_{313}*w_{313} +  x_{323}*w_{323} +  x_{333}*w_{333}) = -25$$

\FloatBarrier
\begin{figure}[htp]
\begin{center}
\includegraphics[scale=0.456]{chap2/c2_figs/8.jpg}
\end{center}
\caption{Thực hiện phép tính convolution trên ảnh màu}
\label{fig:padding,stride}
\end{figure}
\FloatBarrier

Nhận xét:
\begin{itemize}
\item Output Y của phép tính convolution trên ảnh màu là 1 matrix.
\item Có 1 hệ số bias được cộng vào sau bước tính tổng các phần tử của phép tính element-wise.
\end{itemize}

Với mỗi kernel khác nhau ta sẽ học được những đặc trưng khác nhau của ảnh, nên trong mỗi convolutional layer ta sẽ dùng nhiều kernel để học được nhiều thuộc tính của ảnh. Vì mỗi kernel cho ra output là 1 matrix nên k kernel sẽ cho ra k output matrix. Ta kết hợp k output matrix này lại thành 1 tensor 3 chiều có chiều sâu k.

\FloatBarrier
\begin{figure}[htp]
\begin{center}
\includegraphics[scale=0.8]{chap2/c2_figs/15.png}
\end{center}
\caption{Convolutional layer đầu tiên}
\label{fig:conv-firstlayer}
\end{figure}
\FloatBarrier

Output của convolutional layer đầu tiên sẽ thành input của convolutional layer tiếp theo.

Convolutional layer tổng quát
Giả sử input của 1 convolutional layer tổng quát là tensor kích thước $H * W * D$.

Kernel có kích thước $F * F * D$ (kernel luôn có depth bằng depth của input và $F$ là số lẻ), stride: $S$, padding: $P$.

Convolutional layer áp dụng $K$ kernel.
=> Output của layer là tensor 3 chiều có kích thước: $ (\frac{H-F+2P}{S} + 1) * (\frac{W-F+2P}{S} + 1) * K$

\FloatBarrier
\begin{figure}[htp]
\begin{center}
\includegraphics[scale=0.8]{chap2/c2_figs/16.png}
\end{center}
\caption{Convolutional layer tổng quát}
\label{fig:conv-tongquat}
\end{figure}
\FloatBarrier

Lưu ý:
\begin{itemize}
\item Output của convolutional layer sẽ qua hàm activation function trước khi trở thành input của convolutional layer tiếp theo.
\item Tổng số parameter của layer: Mỗi kernel có kích thước $F*F*D$ và có 1 hệ số bias, nên tổng parameter của 1 kernel là $F*F*D + 1$. Mà convolutional layer áp dụng $K$ kernel => Tổng số parameter trong layer này là $K * (F*F*D + 1)$.
\end{itemize}
Mạng Convolution Neural Network, ngoài các lớp Convolution ra còn có các lớp Pooling, Dropout, Dense, và Backnomalization,... để làm cho chúng trở nên "dễ học" hơn.


\subsection{Pooling layer}
\label{ss:Pooling}
Pooling layer thường được dùng giữa các convolutional layer, để giảm kích thước dữ liệu nhưng vẫn giữ được các thuộc tính quan trọng. Kích thước dữ liệu giảm giúp giảm việc tính toán trong model.

Gọi pooling size kích thước $K*K$. Input của pooling layer có kích thước $H*W*D$, ta tách ra làm $D$ ma trận kích thước $H*W$. Với mỗi ma trận, trên vùng kích thước $K*K$ trên ma trận ta tìm maximum hoặc average của dữ liệu rồi viết vào ma trận kết quả. Quy tắc về stride và padding áp dụng như phép tính convolution trên ảnh.

\FloatBarrier
\begin{figure}[htp]
\begin{center}
\includegraphics[scale=0.3]{chap2/c2_figs/9.jpg}
\end{center}
\caption{Phép Pooling với $stride=2, padding=0$}
\label{fig:pooling}
\end{figure}
\FloatBarrier

Nhưng hầu hết khi dùng pooling layer thì sẽ dùng $size=(2,2)$, $stride=2$, $padding=0$. Khi đó output width và height của dữ liệu giảm đi một nửa, depth thì được giữ nguyên.

\FloatBarrier
\begin{figure}[htp]
\begin{center}
\includegraphics[scale=0.5]{chap2/c2_figs/pooling.jpeg}
\end{center}
\caption{Kết quả sau khi qua pooling layer $2*2$.}
\label{fig:pooling}
\end{figure}
\FloatBarrier
\centerline{http://cs231n.github.io/convolutional-networks/}

Có 2 loại pooling layer phổ biến là: max pooling và average pooling.
\FloatBarrier
\begin{figure}[htp]
\begin{center}
\includegraphics[scale=0.5]{chap2/c2_figs/10.jpg}
\end{center}
\caption{Max pooling và average pooling}
\label{fig:pooling}
\end{figure}
\FloatBarrier
Ngoài ra còn có thể dùng convolutional layer với stride > 1 để giảm kích thước dữ liệu thay cho pooling layer.

\subsection{Fully connected layer (Dense layer)}
\label{ss:dense}
Sau khi ảnh được truyền qua nhiều convolutional layer và pooling layer thì model đã học được tương đối các đặc điểm của ảnh (ví dụ mắt, mũi, khung mặt,…) thì tensor của output của layer cuối cùng, kích thước $H*W*D$, sẽ được chuyển về 1 vector kích thước $(H*W*D)$.

\FloatBarrier
\begin{figure}[htp]
\begin{center}
\includegraphics[scale=1]{chap2/c2_figs/17.png}
\end{center}
\caption{Phép Flatten biến tensor về 1 vector}
\label{fig:flatten}
\end{figure}
\FloatBarrier
Sau đó, mỗi điểm của vector sẽ được liên kết với toàn bộ output của mode giống như 1 lớp của mạng Neural Network truyền thống.Và cuối cùng của mạng sẽ có nhiệm vụ phân loại theo như yêu cầu của từng bài toán. Thường sẽ sử dụng hàm softmax để tính đầu ra cho lớp này.




\subsection{Kết luận}
Tổng hợp lại, một mô hình mạng CNN sẽ có cấu trúc chung gần giống như hình \ref{fig:tonghopCNN}

\FloatBarrier
\begin{figure}[htp]
\begin{center}
\includegraphics[scale=1]{chap2/c2_figs/18.png}
\end{center}
\caption{Ví dụ mô hình 1 môn hình convolutional neural network}
\label{fig:flatten}
\end{figure}
\FloatBarrier
\centerline{Input image -> Convolutional layer (Conv) + Pooling layer (Pool) -> Fully connected layer (FC) -> Output}
\centerline{Nguồn: https://www.easy-tensorflow.com/tf-tutorials/convolutional-neural-nets-cnns}

\newpage
\chapter{ƯỚC TÍNH TỌA ĐỘ KHUNG XƯƠNG TỪ ẢNH RGB}
\label{s:pose estimate}
Trong nội dung này, em sẽ trình bày hai phương pháp đề xuất để trích xuất đặc trưng khung xương. Đó là trích xuất đặc trưng từ thiết bị Kinect và trích xuất đặc trưng từ mạng neuron network. Phương pháp trích xuất đặc trưng khung xương từ ảnh RGB qua mạng mobilenet v2 được áp dụng trong đề tài vì tính khả thi, có thể ứng dụng được trong đời sống bằng cách tích hợp lên điện thoại thông minh.

\section{ƯỚC TÍNH DỰA TRÊN CAMERA ĐỘ SÂU KINECT}
\label{ss:kinect}
\subsection{Giới thiệu camera cảm biến độ sâu Kinect của Microsoft}
Kinect là một thiết bị đầu vào,là cảm biến chuyển động do hãng Microsoft sản xuất dành cho Xbox 360 và máy tính Windows. Dựa trên một webcam kiểu add-on ngoại vi cho Xbox 360, nó cho phép người dùng điều khiển và tương tác với Xbox 360 mà không cần phải dùng đến một bộ điều khiển tay cầm, thông qua một giao diện người dùng tự nhiên bằng cử chỉ và lệnh nói. Thiết bị được giới thiệu vào tháng 11 năm 2010 như một phụ kiện của Xbox 360. Cảm biến chiều sâu (depth sensor) được sử dụng trong Kinect được lấy từ việc trích xuất camera hồng ngoại. 

Chức năng chính của Kinect là một công cụ để người dùng tương tác với Xbox 360 bằng cử chỉ và lệnh nói. Vì lý do này, các bộ cảm biến có khả năng thu thập dữ liệu ở độ phân giải 640x480 điểm ảnh. Với các dữ liệu chiều sâu, có thể lấy được "các vector đặc trưng mang hình dáng của khung xương con người(SJM)" của người đứng phía trước của cảm biến. Và với các SJM đó, nó có thể nhận biết được cử chỉ của người sử dụng.

\FloatBarrier
\begin{figure}[htp]
\begin{center}
\includegraphics[scale=0.8]{chap3/c3_figs/kinect.png}
\end{center}
\caption{Camera cảm biến độ sâu Kinect \\Nguồn : \url{https://www.ifixit.com}}
\label{fig:kinect}
\end{figure}
\FloatBarrier

Các thông số cơ bản của cảm biến như sau:
\begin{itemize}
\item Ảnh màu : 1920x1080 @30Hz (15Hz ánh sáng yếu)
\item Ảnh độ sâu : 512x424 @30Hz
\item Tầm xa : 0.5 $\sim$ 4.5m
\item Góc nhìn (Dọc/Ngang) : 70 / 60 độ
\item Số lượng SJM (phát hiện/theo dõi) : 6 / 6 SJMs
\item Số lượng SJM-J : 25 SJM-Js
\item Hệ điều hành : Windows 8/10
\item Cổng tín hiệu : USB 3.0
\end{itemize}

Cơ chế hoạt động: Ban đầu, bộ phần phát tia hồng ngoại sẽ phát ra tia hồng ngoại trong vùng hoạt động của nó. Thông qua phản chiếu các tia hồng ngoại về camera hồng ngoại sẽ thu nhận được các tia phản xạ về. Dựa vào thời gian trễ để đo khoảng cách tới các điểm trong vùng quan sát. Kết quả thu về sẽ là hình ảnh vùng quan sát với những chiều sâu khác nhau.

\FloatBarrier
\begin{figure}[htp]
\begin{center}
\includegraphics[scale=0.8]{chap3/c3_figs/depth.jpg}
\end{center}
\caption{Ảnh độ sâu từ Kinect \\(Nguồn : \url{https://www.zonetrigger.com/)}}
\label{fig:kinect}
\end{figure}
\FloatBarrier

\subsection{Cơ chế trích xuất SJM từ Kinect}
Ưu điểm của việc sử dụng Kinect là các thiết kế phần cứng đã tối ưu cho việc tạo ra SJM, không yêu cầu phải thực hiện lại hoặc tạo ra thuật toán mới cho phần trích xuất đặc trưng. Sơ đồ khối trích xuất SJM của Kinect v2 minh họa bằng hình \ref{figure:TrichKhungXuong}.

\FloatBarrier
\begin{figure}[htp]
\begin{center}
\includegraphics[scale=0.15]{chap3/c3_figs/TrichKhungXuong.png}
\end{center}
\caption{Sơ đồ khối thuật toán trích xuất SJM của thiết bị Kinect \\(Nguồn : "Nhận dạng ngôn ngữ ký hiệu cho người câm" - B.V Phúc, H.C Thịnh)}
\label{fig:TrichKhungXuong}
\end{figure}
\FloatBarrier

\underline{\textbf{Dữ liệu ngõ vào}}

Hành động di chuyển của người trong không gian nhận diện được của thiết bị Kienct v2 (các thông số sẽ được đề cập kỹ hơn trong mục~\ref{GanDacTrung}).\\

\underline{\textbf{Dữ liệu ngõ ra}}
\begin{itemize}
\item Ảnh màu kých thước 1920x1080 pxs.

\item Ảnh biểu diễn độ sâu 512x424 pxs.

\item SJM tương ứng với ảnh hiện tại biễu diễn tọa độ bằng meter (đo tương đối bằng ảnh độ sâu) hoặc px (tương ứng với vị trí của ảnh độ sâu).
\end{itemize}

\section{ƯỚC TÍNH TỪ ẢNH 2D DỰA TRÊN MẠNG NEURAL NETWORK}
\label{ss:2Dpose}
\subsection{Tổng quan phương pháp}
\label{sss:tong_quan_2D_pose}
Phương pháp trích xuất hình dáng khung xương được áp dụng từ bài báo "Realtime Multi-Person 2D Pose Estimation using Part Affinity Fields" \cite{cao2017realtime} sử dụng phương pháp bottom-up ước tính tọa độ khung xương từ ảnh sang không gian 2D. Tuy có rất nhiều mạng pose estimate cả 2D lẫn 3D và cả dense pose(ước tính toàn bộ hình dáng cơ thể người) nhưng phương pháp ước tính 2D được chọn vì tốc độ xử lý có thể đáp ứng realtime và hoạt động trên các thiết bị cấu hình thấp.

Mạng sử dụng  detect tư thế 2D song song của nhiều người trong một ảnh. Phương pháp sử dụng một đại diện không có thông số, PAFs được tham khảo để học cách liên kết các bộ phần cơ thể với mỗi cá nhân trong ảnh. Mô hình mã hóa toàn bộ bối cảnh, cho phép một bước phân tích từ dưới lên trên (bước này có độ chính xác cao, realtime và thực hiện song song nhiều người). Mô hình được thiết kế để kết hợp tìm vị trí các phần và liên kết giữa chúng thông qua 2 nhánh của quá trình dự đoán chuỗi giống nhau. Phương pháp của chúng tối đạt giải nhất cuộc thi COCO 2016 keypoints challenge và vượt trội hơn so với kết quả trước đó trong MPII Multi-Person benchmark về performance và sự hiệu quả.

%\subsection{Kiến trúc mạng}
%\label{sss:method}

\FloatBarrier
\begin{figure}[htp]
\begin{center}
\includegraphics[scale=0.3]{chap3/c3_figs/pipeline.png}
\end{center}
\caption{Tổng thể kiến trúc}
\label{fig:pipelineS}
\end{figure}
\FloatBarrier

Hình 2 mình họa toàn bộ nội dung phương pháp. Hệ thống lấy đầu vào, một ảnh màu có kích thước wxh (hình 2a) và tạo ra ngõ ra, tọa độ của những keyponts cho mỗi cá nhân trong ảnh (hình 2e). Đầu tiên, một mạng CNN đồng thời dự đoán một loạt những confidence maps (cfm) S của những vị trí bộ phận cơ thể (hình 2b) và một loạt những miền vector 2D (vf) L của part affinities, cái mà mã hóa độ liên kết giữa các phần cơ thể (hình 2C). Tập hợp  có J cfm, một map cho mỗi bộ phận, trong đó . Tập hợp  có C vf, một cho mỗi chi, trong đó , mỗi vị trí ảnh trong Lc mã hóa một vector 2D (được show trong hình 1). Cuối cùng, cfm và affinity fields được phân tích bởi suy luận tham lam (hình 2d) để tạo ra các keypoints 2D cho tất cả người trong ảnh.

Phương pháp này đưa toàn bộ ảnh đầu vào qua một mạng CNN 2 nhánh để đồng thời dự đoán những confidence map cho sự detect phần cơ thể, thể hiện trong hình b, và part affinity fields cho sự liên kết các phần, thể hiện trong hình c. Bước phân tích thể hiện một loạt những liên kết giữa hai điểm (liên kết lưỡng cực) để liên kết những phần cơ thể (d). Cuối cùng, chúng tôi lắp ráp chúng lại với nhau tạo thành những tư thế cơ thể hoàn chỉnh cho tất cả những người trong ảnh (e).

\begin{itemize} % chấm đầu dòng
\item Đầu tiên, hình ảnh được truyền qua mạng cơ sở để trích xuất các bản đồ đặc trưng. Trong bài báo, tác giả sử dụng 10 lớp đầu tiên của mô hình VGG-19.
\end{itemize}


\begin{itemize} % chấm đầu dòng
\item Sau đó, các bản đồ tính năng được xử lý với nhiều giai đoạn CNN để tạo: một bộ Bản đồ tin cậy một phần và một bộ các trường có mối quan hệ một phần (PAF)
	\begin{itemize}
	\item \textbf{Confidence Maps} : một bộ bản đồ độ tin cậy 2D S cho các vị trí phần cơ thể. Mỗi vị trí chung có một bản đồ.
	\item \textbf{Part Affinity Fields} : một tập hợp các trường vectơ 2D L mã hóa mức độ liên kết giữa các phần.
	\end{itemize}
\end{itemize}


\begin{itemize} % chấm đầu dòng
\item Cuối cùng, \textbf{Confidence Maps} và \textbf{Part Affinity Fields} được xử lý bằng thuật toán tham lam để có được tư thế cho mỗi người trong ảnh.
\end{itemize}

\subsection{Giải thích các phần}
%\renewcommand{\labelitemi}{$\square$}
%\renewcommand\labelitemii{$\nabla$}
%\renewcommand\labelitemiii{$\square$}
\begin{itemize}
  \item[$\square$] \textbf{Confidence Maps}
  Confidence Maps là một đại diện 2D cho niềm tin rằng một bộ phận cơ thể cụ thể có thể được đặt trong bất kỳ pixel nào. Với J là số lượng vị trí bộ phận cơ thể (khớp). Sau đó, \textbf{Confidence Maps} $S = (S_1, S_2, .., S_J)$ với $S_j \in R^{w \times h},j \in (1 \ldots J)$
  Tóm lại, mỗi bản đồ tương ứng với một khớp và có cùng kích thước với hình ảnh đầu vào .

  \item[$\square$] \textbf{Part Affinity Fields(PAF)}
  Trường quan hệ một phần \textbf{(PAF)} là một tập hợp các trường dòng mã hóa các mối quan hệ cặp đôi không cấu trúc giữa các bộ phận cơ thể.

Mỗi cặp bộ phận cơ thể có một \textbf{PAF} , tức là cổ, mũi, khuỷu tay, v.v.

Cho $C$ là số lượng các cặp phần trên cơ thể. Sau đó \textbf{PAFs} là các thiết lập \textbf{$L = (L_1, L_2, ..., L_c)$} với \textbf{$L_c \in R^{w \times h \times 2},c \in (1 \ldots C)$}

Nếu một pixel nằm trên một chi (phần cơ thể), giá trị trong $L_c$ tại pixel đó là một vectơ đơn vị 2D từ khớp bắt đầu đến khớp cuối.
\item[$\square$] \textbf{CNN nhiều giai đoạn}

\FloatBarrier
\begin{figure}[htp]
\begin{center}
\includegraphics[scale=0.65]{chap3/c3_figs/CNN_m.png}
\end{center}
\caption{Kiến trúc của CNN nhiều giai đoạn từ phiên bản tạp chí của OpenPose}
\label{fig:pipelineS}
\end{figure}
\FloatBarrier

\textbf{CNN nhiều giai đoạn gồm các bước như sau:}
\begin{itemize}
\item Tính toán các \textbf{part affinity fields (PAFs)}, $ L^{1}$ từ feature maps của mạng cơ sở $F$. Cho $\phi^1$ là mạng CNN mạng CNN tại bước 1. 
$$L^1 = \phi^1(F) $$

\item Giai đoạn $t$ đến giai đoạn $T_P$: Tinh chỉnh dự đoán của \textbf{PAF} từ giai đoạn trước bằng cách sử dụng bản đồ tính năng $F$ và các \textbf{PAF} trước đó $(L^{t-1})$.Với $\phi^t$ là CNN ở giai đoạn t.
$$L^t = \phi^t(F, L^{t-1}), \forall 2 \leq t \leq T_P$$
\item Sau khi $T_P$ lặp đi lặp lại, quá trình được lặp lại việc phát hiện \textbf{confidence maps} , bắt đầu trong dự đoán PAF cập nhật mới nhất . $\rho^t$ là CNN ở giai đoạn $t$. Quá trình được lặp lại cho $T_C$.
$$S^{T_P} = \rho^t(F, L^{T_P}), \forall t = T_P$$
$$S^t = \rho^t(F, L^{T_P}, S^{t-1}), \forall T_P \leq t \leq T_P + T_C$$

\item Ma trận $S$ và $L$ cuối cùng là \textbf{Confidence Maps} và \textbf{part affinity fields (PAFs)} sẽ được xử lý thêm bằng thuật toán tham lam.

\end{itemize}
\end{itemize}
\textbf{Chú thích:}
CNN nhiều giai đoạn này là từ phiên bản tạp chí 2018. Trong phiên bản CVPR 2017 gốc, họ đã tinh chỉnh cả bản đồ độ tin cậy và các trường mối quan hệ một phần (PAF) ở mỗi giai đoạn. Do đó, họ đòi hỏi nhiều tính toán và thời gian hơn ở mỗi giai đoạn. Trong cách tiếp cận mới, tác giả nhận thấy rằng cách tiếp cận mới làm tăng cả tốc độ và độ chính xác tương ứng 200\% và 7\%.

\subsection{Kiến trúc mạng}
\label{sss:structure}
\subsubsection{Phát hiện và liên kết đồng thời}

Kiến trúc của mạng, được show trong hình \ref{fig:structure}, dự đoán đồng thời những cfm và affinity fields \textbf{(af)} cái mà mã hóa liên kết giữa các phần. Mạng được tách thành 2 nhánh: Nhánh trên, màu hường da, dự đoán những \textbf{cfm}, và nhánh dưới, màu xanh, dự đoán những \textbf{af}. Mỗi nhánh là một kiến trúc dự đoán lặp lại, theo như \ref{wei2016convolutional}, nó tinh chỉnh những dự đoán qua các giai đoạn liên tiếp $t \in {{1 \ldots T}}$,  với giám sát trung gian ở từng giai đoạn.

\FloatBarrier
\begin{figure}[htp]
\begin{center}
\includegraphics[scale=0.45]{chap3/c3_figs/structure.png}
\end{center}
\caption{Kiến trúc của mạng CNN nhiều bước 2 nhánh. Mỗi bước trong nhánh đầu tiên dự đoán những $CFM S^{t}$, và mỗi bước trong nhánh thứ 2 dự đoán $PAFs L^{t}$. Sau mỗi bước, những dự đoán từ 2 nhánh, cùng với những đặc trưng ảnh, được nối lại cho bước tiếp theo.}
\label{fig:structure}
\end{figure}
\FloatBarrier

\subsection{Phân tích nhiều người sử dụng PAFs}
\label{ss:Multi-Person Parsing using PAFs}

Trong phần này, sẽ nói về tổng quan về thuật toán tham lam được sử dụng để phân tích tư thế của nhiều người từ PAFs và CFMs.

Quá trình phân tích được tóm tắt thành 3 bước sau:
\begin{itemize}
\item Bước 1: Tìm tất cả những vị trí khớp sử dụng confidence maps.
\item Bước 2: Tìm những khớp cùng nhau tạo các chi (bộ phận cơ thể) sử dụng PAFs và những khớp trong bước 1.
\item Bước 3: Kết nối các chi thuộc cùng một người và tạo danh sách những tư thế cơ thể.
\end{itemize}

\subsubsection{Bước 1 - Tìm tất cả những vị trí khớp sử dụng confidence maps.}
\begin{itemize}
\item \textbf{Đầu vào:}
	\begin{itemize}
	\item Confidence maps, $S = (S_1, S_2, .., S_J)$ trong đó $S_j \in R^{w \times h},j \in (1 \ldots J)$.
	\item Up-sampling scale: Sự khác biệt giữa dài/rộng giữa ảnh đầu vào và confidence maps.
	\end{itemize}
\item \textbf{Đầu ra: }
	%\begin{itemize}
	%\item Danh sách khớp(joints_list): một danh sách những vị trí khớp của kích thước J, mỗi phần tử là một danh sách các đỉnh (x, y, xác suất).
	%\item Ví dụ, kích thước của joints_list là 18 cho 18 vị trí khớp (mũi, cổ,...) và những phần tử trong joints_list là những danh sách có độ dài khác nhau. Những phần tử này lưu trữ thông tin đỉnh (vị trí x, y và điểm xác suất) cho mỗi vị trí khớp.
	%\end{itemize}
\item \textbf{Xử lý: Cho mỗi khớp từ 1 dến J:}
	\begin{itemize}
	\item Lấy heatmap 2D tương ứng cho khớp trong confidence maps.
	\item Tìm những đỉnh bằng cách lấy ngưỡng heatmap 2D.
	\item Đối với mỗi đỉnh:
		\begin{itemize}
		\item Lấy một đốm xung quanh đỉnh trong heap.
		\item Phóng to đốm sử dụng up-sampling scale.
		\item Lấy vị trí đỉnh lớn nhất trong đốm được phóng to.
		\item Thêm thông tin đỉnh vào danh sách các đỉnh của khớp.
		\end{itemize}
	\end{itemize}
\end{itemize}

\subsubsection{Bước 2 - Tìm những khớp cùng nhau tạo các chi (bộ phận cơ thể) sử dụng PAFs và những khớp trong bước 1.}
\begin{itemize}
\item \textbf{Đầu vào:}
	%\begin{itemize}
	%\item joints_list ngõ ra của bước 1.
	%\item \textbf{PAFs}: \textbf{$L = (L_1, L_2, ..., L_c)$} trong đó \textbf{$L_c \in R^{w \times h \times 2},c \in (1 \ldots C)$}

	%\end{itemize}
%\end{itemize}
\item \textbf{Đầu ra: }
	%\begin{itemize}
	%\item Danh sách khớp (joints_list): một danh sách những vị trí khớp của kích thước J, mỗi phần tử là một danh sách các đỉnh (x, y, xác suất). Ví dụ, kích thước của joints_list là 18 cho 18 vị trí khớp (mũi, cổ,...) và những phần tử trong joints_list là những danh sách có độ dài khác nhau. Những phần tử này lưu trữ thông tin đỉnh (vị trí x, y và điểm xác suất) cho mỗi vị trí khớp.
	%\end{itemize}
\item \textbf{Xử lý: Cho mỗi khớp từ 1 dến J:}
	\begin{itemize}
	\item Lấy heatmap 2D tương ứng cho khớp trong confidence maps.
	\item Tìm những đỉnh bằng cách lấy ngưỡng heatmap 2D.
	\item Đối với mỗi đỉnh:
		\begin{itemize}
		\item Lấy một đốm xung quanh đỉnh trong heap.
		\item Phóng to đốm sử dụng up-sampling scale.
		\item Lấy vị trí đỉnh lớn nhất trong đốm được phóng to.
		\item Thêm thông tin đỉnh vào danh sách các đỉnh của khớp.
		\end{itemize}
	\end{itemize}
\end{itemize}

\subsubsection{Bước 3 - Kết nối các chi thuộc cùng một người và tạo danh sách những tư thế cơ thể.}

\newpage
\chapter{ĐÊ XUẤT: NHẬN DẠNG NGÔN NGỮ KÝ HIỆU TỪ TỌA ĐỘ KHUNG XƯƠNG BẰNG MẠNG DNN}
\label{s:DNN}
Trong chương 3, thuật toán ước tính tư thế khung xương đã được trình bày chi tiết về lý thuyết và phương pháp xử lý áp dụng thuật toán. Ta thấy rằng phương pháp này hỗ trợ rất tốt cho việc ước tính nh anhtư thế khung xương trong xử lý thời gian thực. Trong chương 4 này, luận văn sẽ trình bày về đề xuất mô hình mạng DNN lấy đầu vào là toạ độ khung xương được ước tính và nhận dạng cử chỉ thời gian thực. Nội dụng chương trình bày cấu trúc mạng neural network đề xuất, cách thu thập và xử lý dữ liệu cũng như sơ đồ hoạt động chương trinh.
\section{Tổng quan}
Một hành động của con người được đánh giá, xem xét bằng một loạt các cử chỉ theo thời gian. Tuy nhiên, khi xem xét các hành động của con người nhằm tìm ra một cấu trúc nhất quán và có thể tạo thành mô hình thì gặp các vấn đề phức tạp sau:

$\bullet$ Nếu chỉ xem xét đường bao của con người, các phần cơ thể của con người quá gần nhau để có thể xác định được chính xác phần cơ thể cần thiết.

$\bullet$ Hình dạng của cử chỉ (đường bao con người, màu sắc các phần cơ thể), vị trí, loại và kiểu của cử chỉ rất phức tạp. Việc xem xét một mô hình có thể biểu diễn toàn bộ các hành động ngôn ngữ ký hiệu hầu như không thể xem xét nên trong phạm vi luận văn này chỉ xem xét đến việc một mô hình có thể biểu diễn 16 cử chỉ cần sự phối hợp cả hai tay và các cử chỉ tương đối khác nhau (\textbf{Xin chào, Tôi, thành phố, vui vẻ, ẵm em, Sài Gòn, Vĩnh Long, đi bộ, mùa màng, đói bụng, yêu, ăn, biểu quyết, đứng yên, hẹp, rộng}).

$\bullet$ Các hành động của con người có thể giống nhau, tuy nhiên nếu việc quan sát hoặc camera quan sát nằm ở vị trí khác nhau, hướng, độ cao,... đều ảnh hưởng đến khả năng nhận diện hành động của con người. Luận văn đã nêu được phương pháp để có thể phát triển cho việc xác định hành động của con người khi vị trí của camera thay đổi. Tuy nhiên việc kiểm chứng khả năng hoạt động ở các vị trí camera khác nhau sẽ được xem xét ở tương lai.

$\bullet$ Nếu một phần cơ thể bị che khuất bởi các vật thể thì việc xác định hành động của con người sẽ gặp khó khăn hơn rất nhiều so với trường hợp không bị che khuất, phạm vi luận văn không xem xét đến vấn đề này. Tuy nhiên đây là một điểm cần xem xét đến để có thể hoàn thiện hệ thống nhận diện cử chỉ trong tương lai.
\section{Thu thập dữ liệu}
Dữ liệu đầu vào là tọa độ của 18 khớp xương được detect từ mạng mobilenet. Các khớp xương được xuất ra từ mạng được đánh số thứ tự từ 0 tới 17. Các khớp xương cụ thể được thể hiện trong bảng \ref{table:joints} và trong hình \ref{fig:joints}.

\FloatBarrier
\begin{table}[h]
\caption{Các khớp xương được xuất ra từ mạng}
\label{table:joints}
\centering
\begin{center}
\begin{tabular}{|c|p{9cm}|} 
 \hline
Số thứ tự khớp xương  & Vị trí \\
 \hline
 0 & Mũi\\
 \hline 
 1 & Cổ\\
 \hline 
 2 & Vai phải\\
 \hline
 3 & Cùi chỏ phải \\
 \hline 
 4 & Cổ tay phải\\
 \hline
 5 & Vai trái\\
 \hline
 6 & Cùi chỏ trái\\
 \hline
 7 & Cổ tay trái\\
 \hline
 8 & Hông phải\\
 \hline
 9 & Đầu gối phải\\
 \hline
 10 & Cổ chân phải\\
 \hline
 11 & Hông trái\\
 \hline
 12 & Đầu gối trái\\
 \hline
 13 & Cổ chân trái\\
 \hline
 14 & Mắt phải\\
 \hline
 15 & Mắt trái\\
 \hline
 16 & Tai phải\\
 \hline
 17 & Tai trái\\
 \hline
\end{tabular}
\end{center}
\end{table}
\FloatBarrier

\FloatBarrier
\begin{figure}[htp]
\begin{center}
\includegraphics[scale=0.6]{chap4/c4_figs/joints_order.png}
\end{center}
\caption{Sơ đồ khớp xương xuất ra từ mạng mobilenet}
\label{fig:joints}
\end{figure}
\FloatBarrier

Ngôn ngữ ký hiệu với đặc trưng là phần trên cơ thể xuất hiện 
\section{Xử lý dữ liệu đầu vào}
\subsection{Loại bỏ các phần SJM dư thừa}

\subsection{Chuẩn hóa SJM để phân loại đặc trưng}
\FloatBarrier
\begin{figure}[htp]
\begin{center}
\includegraphics[scale=0.25]{chap4/c4_figs/datajoint.png}
\end{center}
\caption{Dữ liệu khớp xương sau khi đã loại bỏ hết các phần không cần thiết}
\label{fig:pipelineS}
\end{figure}
\FloatBarrier

\section{Cấu trúc mạng neural network đề xuất}

\section{Huấn luyện mạng}

\FloatBarrier
\begin{figure}[htp]
\begin{center}
\includegraphics[scale=1]{chap4/c4_figs/train_val_acc.png}
\end{center}
\caption{Accurracy của tập train và validate}
\label{fig:pipelineS}
\end{figure}
\FloatBarrier

\FloatBarrier
\begin{figure}[htp]
\begin{center}
\includegraphics[scale=1]{chap4/c4_figs/train_val_l.png}
\end{center}
\caption{Loss của tập train và validate}
\label{fig:pipelineS}
\end{figure}
\FloatBarrier

\section{Kết quả}

\FloatBarrier
\begin{figure}[htp]
\begin{center}
\includegraphics[scale=1]{chap4/c4_figs/confusion_matrix.png}
\end{center}
\caption{Confusion matrix của 16 lớp phân loại}
\label{fig:pipelineS}
\end{figure}
\FloatBarrier

\newpage
\chapter{Giải thuật Deep Sort theo dõi từng người trong khung hình}
Qua các chương trước, luận văn đã trình bày thuật toán, lý thuyết cũng nhhư cách thức hoạt động của chương trình nhận dạng ngôn ngữ ký hiệu. Ở chương này, luận văn sẽ trình bày giải thuật Deep Sort dùng để theo dõi từng người trong khung hình, đánh số thứ tự từng người, cùng với ngôn ngữ ký hiệu của họ muốn diễn đạt.

Phát hiện đối tượng và theo dõi đối tượng là một trong những chủ đề được thị giác máy tính nghiên cứu từ rất lâu. Các thành tựu của chúng đã đạt đến những thành công rất cao cũng như được ứng dụng rộng rãi vào đời sống. Phát hiện đối tượng (objects detection) chỉ tập trung vào việc phát hiện từng đối tượng, đặt chúng trong từng khung hình riêng lẻ và sau đó phân loại đối tượng đó. Việc phát hiện đối tượng chỉ dừng lại ở đây, như vậy đối với cùng một đối tượng nhưng với các khung hình liên tiếp nhau, máy tính sẽ không thể biết được 2 khung hình này chứa cùng một đối tượng . Khác với các thuật toán phát hiện đối tượng, các thuật toán theo dõi đối tượng đều hoạt động theo cách thức khóa từng đối tượng trong khung hình, xác định duy nhất từng đối tượng và theo dõi tất cả chúng cho đến khi chúng rời khỏi khung hình. Ví dụ, nếu máy tính phát hiện được 3 ôtô trong một khung hình, trình theo dõi sẽ phải xác định được 3 đối tượng riêng biệt, đánh số thứ tự chúng và theo dõi qua các khung hình tiếp theo. Theo dõi đối tượng bao gồm theo dõi đối tượng đơn(single object tracking) và theo dõi đa đối tượng (multiple object tracking). Trong phần này, luận văn sẽ chủ yếu đề cập đến các thuật toán theo dõi đa đối tượng cụ thể là giải thuật Deep Sort (được ứng dụng trong luận văn).  
 

\section{Giải thuật Deep Sort}

Phổ biến nhất và là một trong những thuật toán đơn giản nhất để theo dõi là Deep Sort (Simple Online and Realtime Tracking \cite{bewley2016simple}). Thuật toán này có thể theo dõi nhiều đối tượng trong thời gian thực nhưng thuật toán chỉ liên kết các đối tượng đã phát hiện trên các khung khác nhau dựa trên tọa độ của kết quả phát hiện, như hình \ref{fig:sort}

\FloatBarrier
\begin{figure}[htp]
\begin{center}
\includegraphics[scale=0.3]{chap5/c5_figs/sort.png}
\end{center}
\caption{Thuật toán Sort\\Nguồn : "Simple Online and Realtime Tracking with a Deep Association Metric" \cite{wojke2017simple}}
\label{fig:sort}
\end{figure}
\FloatBarrier

\section{Kết quả}
Áp dụng giải thuật Deep Sort, luận văn đã theo dõi và đánh số thứ tự được những người có trong khung hình.
\FloatBarrier
\begin{figure}[htp]
\begin{center}
\includegraphics[scale=0.8]{chap5/c5_figs/deep_sort.png}
\end{center}
\caption{Kết quả Deep Sort}
\label{fig:deep_sort}
\end{figure}
\FloatBarrier

\newpage
\chapter{THỰC NGHIỆM KẾT QUẢ VÀ ĐÁNH GIÁ}
Qua các chương trước, luận văn đã trình bày thuật toán, lý thuyết cũng như cách thức hoạt động của chương trình nhận dạng ngôn ngữ ký hiệu. Ở chương này, luận văn sẽ trình bày việc xây dựng ứng dụng nhận diện dựa trên mô hình đã huấn luyện, kết quả thực nghiệm mà luận văn đã đạt được sau khi huấn luyện và các phương pháp đánh giá độ chính xác của mô hình.
\section{Thực hiện phần mềm nhận dạng ngôn ngữ ký hiệu}
Tiếp theo, luận văn tiến hành xây dựng phần mềm nhận dạng thời gian thực để ứng dụng được mô hình đã huấn luyện trên ra thực tế. Việc xây dựng phần mềm ứng dụng này giúp người sử dụng có thể tương tác với ứng dụng thông qua nút bấm và chuột, làm chúng thân thiện với người dùng hơn.
\begin{itemize}
\item[$\square$] \textbf{Ngôn ngữ lập trình: }
Python được sử dụng làm ngôn ngữ lập trình chính để xây dựng model cũng như viết ứng dụng nhận dạng. Python là ngôn ngữ lập trình hướng đối tượng, cấp cao, mạnh mẽ, được tạo ra bởi Guido van Rossum. Python được sử dụng nhiều trong machine learning và deep learning bởi vì tính phổ biến và có rất nhiều thư viện hỗ trợ từ việc xử lý dữ liệu cho đến xây dựng mô hình machine learning như Tensorflow, Keras, Pandas, ...
\item[$\square$] \textbf{Thư viện Keras}
Thư viện Keras được sử dụng trong việc xây dựng model trong luận văn. Keras là một framework mã nguồn mở cho deep learning được viết bằng Python. Nó có thể chạy trên nền của các deep learning framework khác như: tensorflow, theano, CNTK. Với các API bậc cao, dễ sử dụng, dễ mở rộng, keras giúp người dùng xây dựng các deep learning model một cách đơn giản. Keras có một số ưu điểm như :

	\begin{itemize}
	\item Dễ sử dụng,xây dựng model nhanh.
	\item Có thể run trên cả cpu và gpu
	\item Hỗ trợ xây dựng CNN , RNN và có thể kết hợp cả 2.
	\end{itemize}
\end{itemize}

\FloatBarrier
\begin{figure}[htp]
\begin{center}
\includegraphics[scale=0.5]{chap6/c6_figs/keras.png}
\end{center}
\caption{Thư viện hỗ trợ Keras}
\label{fig:keras}
\end{figure}
\FloatBarrier

Chương trình nhận dạng được xây dựng trên hệ điều hành ubuntu, sử dụng camera là webcam. Giao diện được viết trên thư viện PyQt5, là một thư viện hỗ trợ viết giao diện ứng dụng trên ngôn ngữ Python. Các nút điều khiển trên giao diện, giúp chọn các chế độ hoạt động của chương trình. Có tất cả 3 chế độ đều có thể hoạt động khi có nhiều người trong khung hình:
\begin{itemize}
\item Skeleton detection: Phát hiện và và trích đặc trưng khung xương từ camera.
\item Tracking: Theo dõi và đánh chỉ số của từng người trong khung hình dựa trên giải thuật Deep Sort.
\item Recognize: Theo dõi, đánh chỉ số từng người và nhận dạng ngôn ngữ ký hiệu.
\end{itemize}

\section{Kết quả}
Chương trình thực nghiệm đã nhận dạng được khá chính xác các hành động đã huấn luyện (16 ngôn ngữ ký hiệu) khi xử lý realtime. Tốc độ xử lý khá nhanh khi đạt được từ 8 -> 10 FPS. Giao diện và các mode hoạt động được trình bày như các hình \ref{fig:gui1}, \ref{fig:gui2}, \ref{fig:gui3}.

\FloatBarrier
\begin{figure}[htp]
\begin{center}
\includegraphics[scale=0.5]{chap6/c6_figs/mode_skeleton.png}
\end{center}
\caption{Giao diện phần mềm nhận dạng - Mode "Skeleton detection"}
\label{fig:gui1}
\end{figure}

\begin{figure}[htp]
\begin{center}
\includegraphics[scale=0.5]{chap6/c6_figs/mode_tracking.png}
\end{center}
\caption{Giao diện phần mềm nhận dạng - Mode "Tracking"}
\label{fig:gui2}
\end{figure}

\begin{figure}[htp]
\begin{center}
\includegraphics[scale=0.5]{chap6/c6_figs/mode_recognize.png}
\end{center}
\caption{Giao diện phần mềm nhận dạng - Mode "Recognize"}
\label{fig:gui3}
\end{figure}
\FloatBarrier
 
Kết quả khi xử lý với nhiều người trong khung hình:

\begin{figure}[htp]
\begin{center}
\includegraphics[scale=0.5]{chap6/c6_figs/kq.png}
\end{center}
\caption{Kết quả nhận dạng đối với nhiều người}
\label{fig:kq}
\end{figure}
\FloatBarrier

\section{Đánh giá}
\subsection{Đánh giá mô hình nhận dạng ngôn ngữ ký hiệu}
Khi xây dựng một mô hình Machine Learning, chúng ta cần một phép đánh giá để xem mô hình sử dụng có hiệu quả không và để so sánh khả năng của các mô hình. Trong phần này, luận văn sẽ trình bày các mô hình classification.

Hiệu năng của một mô hình thường được đánh giá dựa trên tập dữ liệu kiểm thử (test data). Cụ thể, giả sử đầu ra của mô hình khi đầu vào là tập kiểm thử được mô tả bởi vector y\_pred - là vector dự đoán đầu ra với mỗi phần tử là class được dự đoán của một điểm dữ liệu trong tập kiểm thử. Ta cần so sánh giữa vector dự đoán y\_pred này với vector class thật của dữ liệu, được mô tả bởi vector y\_true. Đối với bài toán phân loại của luận văn, có 16 lớp dữ liệu được gán nhãn là tên các từ được huấn luyện.


Có rất nhiều cách đánh giá một mô hình phân lớp. Tuỳ vào những bài toán khác nhau mà chúng ta sử dụng các phương pháp khác nhau. Các phương pháp thường được sử dụng là: accuracy score, confusion matrix, ROC curve, Area Under the Curve, Precision and Recall, F1 score, Top R error, ...

Trong phần này, luận văn sẽ trình bày về phương pháp đánh giá sử dụng accuracy và confusion matrix. Sau đó đánh giá mô hình của luận văn trên tập dữ liệu test data gồm gần 400 SJM đối với mỗi từ.

\subsubsection{Accuracy}
Cách đơn giản và hay được sử dụng nhất là accuracy (độ chính xác). Cách đánh giá này đơn giản tính tỉ lệ giữa số điểm được dự đoán đúng và tổng số điểm trong tập dữ liệu kiểm thử.

Với bộ test data, model đạt được accuracy = 0.9961, cho thấy mô hình đã không bị overfiting và có khả năng áp dụng vào thực tế.
\subsubsection{Confusion Matrix}
Việc sử dụng accuracy như trên chỉ cho chúng ta biết được bao nhiêu phần trăm lượng dữ liệu được phân loại đúng mà không chỉ ra được cụ thể mỗi loại được phân loại như thế nào, lớp nào được phân loại đúng nhiều nhất, và dữ liệu thuộc lớp nào thường bị phân loại nhầm vào lớp khác. Ngoài ra đối với các bài toán thực tế nếu số lượng dữ liệu bị mất cân bằng giữa các lớp thì đại lượng accuracy chưa đủ ý nghĩa. Để có thể đánh giá được các giá trị này, chúng ta sử dụng một ma trận được gọi là confusion matrix.

Về cơ bản, confusion matrix thể hiện bao nhiêu điểm thực sự thuộc vào một class, và được dự đoán là rơi vào một class. Confusion matrix là một ma trận vuông với kích thước mỗi chiều bằng số lượng các lớp dữ liệu. Giá trị tại hàng thứ $i$, cột thứ $j$ là số lượng điểm lẽ ra thuộc class $i$ nhưng được dự đoán vào class $j$. Tổng các phần tử trong toàn ma trận chính là số điểm trong tập kiểm thử. Các phần tử trên đường chéo của ma trận là số điểm được phân loại đúng của mỗi lớp dữ liệu. Từ đây có thể suy ra accuracy chính bằng tổng các phần tử trên đường chéo chia cho tổng các phần tử của toàn ma trận. Đối với bài toán trong luận văn, confusion matrix được áp dụng trên tập dữ liệu test ta được kết quả như hình \ref{fig:confusion_matrix}.
\newpage
\FloatBarrier
\begin{figure}[htp]
\begin{center}
\includegraphics[scale=0.25]{chap6/c6_figs/confusion_matrix1.png}
\end{center}
\caption{Confusion matrix của 16 lớp phân loại}
\label{fig:confusion_matrix}
\end{figure}
\FloatBarrier

Ta thấy rằng, confusion matrix mang nhiều thông tin hơn, giúp chúng ta xác định cụ thể model nhận diện kết quả với từng lớp như thế nào. Dựa vào confusion matrix \ref{fig:confusion_matrix} ta thấy mô hình đã nhận diện chính xác cao đối với từng lớp. Kết quả mô hình đã đạt được mục tiêu ban đầu của luận văn.
\\[1cm]
\subsection{Đánh giá khả năng hoạt động của ứng dụng}
Ứng dụng sau khi được xây dựng hoàn thiện, các thử nghiệm được tiến hành để đánh giá độ chính xác, khả năng nhận diện trong các điều kiền khác nhau của mô hình. Ứng dụng đánh giá được chạy trên máy tính có cấu hình phần cứng như sau:

\FloatBarrier
\begin{table}[h]
\centering
\captionsetup{list=no}
\begin{center}
\begin{tabularx}{\columnwidth}{XX}
Tên máy & : ROG-Strix-G531  \\
Chip & : $Intel^{\circledR } Core^{TM} i7-9750 CPU 2.60GHz$ x 12 \\
Card đồ họa & : NVIDIA GeForce GTX1650 \\
Hệ điều hành & : Ubuntu 18.04 64-bit \\
Camera & : Logitech C270 (720p/30fps)
\end{tabularx}
\end{center}
\end{table}
\FloatBarrier

Sau các thử nghiệm của ứng dụng trên nhiều điều kiện khác nhau, luận văn có một số điểm rút ra như sau:

\begin{itemize}
\item[$\square$] Đối với thử nghiệm về khoảng cách, khoảng cách giữa người dùng với camera được thay đổi từ 0.5(m) $\rightarrow$ 10(m). Thử nghiệm này cho thấy:
	\begin{itemize}
	\item Ứng dụng nhận dạng hoạt động tốt nhất khi khoảng cách người tới camera là từ 2 $\rightarrow$ 4(m).
	\item Khi khoảng cách quá gần (< 1m), các khớp xương cần thiết để nhận diện từ ngữ ký hiệu hoàn toàn bị khuất ra khỏi khung hình do vậy sẽ không thể nhận diện được.
	\item Khi khoảng cách quá xa (> 10m), hình ảnh con người sẽ nhỏ do vậy mô hình dự đoán sẽ không thể xác định được hết các khớp hoặc dự đoán sai lệch đi. Các ký hiệu khó dự đoán chính xác nhất khi khoảng cách xa là : "Sài Gòn", "Vĩnh Long" và "ăn".
	\end{itemize}
\end{itemize}

\begin{itemize}
\item[$\square$] Đối với thử nghiệm về số lượng người trong khung hình, số lượng người thử nghiệm được thay đổi trong khung hình từ 0 $\rightarrow$ 3(người):
	\begin{itemize}
	\item Khi không có người trong hình, FPS của ứng dụng đạt ~ 30 fps(gần bằng fps của camera).
	\item Khi có một người trong hình, FPS ứng dụng đạt 4 ~ 8(fps).
	\item Khi có nhiều hơn 3 người trong hình, fps đạt dưới 3.
	\end{itemize}
\end{itemize}

\newpage
\chapter{KẾT LUẬN VÀ HƯỚNG PHÁT TRIỂN}
\section{Kết luận}
\label{ss:ket_luan}

\section{Hướng phát triển}



\newpage
\addcontentsline{toc}{chapter}{PHỤ LỤC}
\thispagestyle{phuluc}
\fontsize{25pt}{40pt}{\selectfont{Phụ lục}}

\fontsize{18pt}{30pt}{\selectfont{\textbf{Các từ nhận dạng được:}}}

\begin{itemize}

\item \textbf{Xin chào}
\FloatBarrier
\begin{figure}[htp]
\begin{center}
\includegraphics[scale=0.4]{kq/xin_chao.png}
\end{center}
\caption{Ký hiệu "xin chào"}
\end{figure}
\FloatBarrier

\item \textbf{Tôi}
\FloatBarrier
\begin{figure}[htp]
\begin{center}
\includegraphics[scale=0.4]{kq/toi.png}
\end{center}
\caption{Ký hiệu "Tôi"}
\end{figure}
\FloatBarrier

\thispagestyle{phuluc}
\pagebreak

\item \textbf{Thành phố}
\FloatBarrier
\begin{figure}[htp]
\begin{center}
\includegraphics[scale=0.4]{kq/thanh_pho.png}
\end{center}
\caption{Ký hiệu "thành phố"}
\end{figure}
\FloatBarrier

\item \textbf{Vui vẻ}
\FloatBarrier
\begin{figure}[htp]
\begin{center}
\includegraphics[scale=0.4]{kq/vui_ve.png}
\end{center}
\caption{Ký hiệu "vui vẻ"}
\end{figure}
\FloatBarrier

\thispagestyle{phuluc}
\pagebreak

\item \textbf{Ẵm em}
\FloatBarrier
\begin{figure}[htp]
\begin{center}
\includegraphics[scale=0.4]{kq/am_em.png}
\end{center}
\caption{Ký hiệu "ẵm em"}
\end{figure}
\FloatBarrier

\item \textbf{Sài Gòn}
\FloatBarrier
\begin{figure}[htp]
\begin{center}
\includegraphics[scale=0.4]{kq/sai_gon.png}
\end{center}
\caption{Ký hiệu "Sài Gòn"}
\end{figure}
\FloatBarrier

\thispagestyle{phuluc}
\pagebreak

\item \textbf{Đi bộ}
\FloatBarrier
\begin{figure}[htp]
\begin{center}
\includegraphics[scale=0.4]{kq/di_bo.png}
\end{center}
\caption{Ký hiệu "đi bộ"}
\end{figure}
\FloatBarrier

\item \textbf{Mùa màng}
\FloatBarrier
\begin{figure}[htp]
\begin{center}
\includegraphics[scale=0.4]{kq/mua_mang.png}
\end{center}
\caption{Ký hiệu "mùa màng"}
\end{figure}
\FloatBarrier

\thispagestyle{phuluc}
\pagebreak

\item \textbf{Đói bụng}
\FloatBarrier
\begin{figure}[htp]
\begin{center}
\includegraphics[scale=0.4]{kq/doi_bung.png}
\end{center}
\caption{Ký hiệu "đói bụng"}
\end{figure}
\FloatBarrier

\item \textbf{Yêu}
\FloatBarrier
\begin{figure}[htp]
\begin{center}
\includegraphics[scale=0.4]{kq/yeu.png}
\end{center}
\caption{Ký hiệu "Yêu"}
\end{figure}
\FloatBarrier

\thispagestyle{phuluc}
\pagebreak

\item \textbf{Ăn}
\FloatBarrier
\begin{figure}[htp]
\begin{center}
\includegraphics[scale=0.4]{kq/an.png}
\end{center}
\caption{Ký hiệu "ăn"}
\end{figure}
\FloatBarrier

\item \textbf{Biểu quyết}
\FloatBarrier
\begin{figure}[htp]
\begin{center}
\includegraphics[scale=0.4]{kq/bieu_quyet.png}
\end{center}
\caption{Ký hiệu "biểu quyết"}
\end{figure}
\FloatBarrier

\thispagestyle{phuluc}
\pagebreak

\item \textbf{Đứng yên}
\FloatBarrier
\begin{figure}[htp]
\begin{center}
\includegraphics[scale=0.4]{kq/dung_yen.png}
\end{center}
\caption{Ký hiệu "đứng yên"}
\end{figure}
\FloatBarrier

\item \textbf{Hẹp}
\FloatBarrier
\begin{figure}[htp]
\begin{center}
\includegraphics[scale=0.4]{kq/hep.png}
\end{center}
\caption{Ký hiệu "hẹp"}
\end{figure}
\FloatBarrier

\thispagestyle{phuluc}
\pagebreak

\item \textbf{Rộng}
\FloatBarrier
\begin{figure}[htp]
\begin{center}
\includegraphics[scale=0.4]{kq/rong.png}
\end{center}
\caption{Ký hiệu "rộng"}
\end{figure}
\FloatBarrier

\item \textbf{Vĩnh Long}
\FloatBarrier
\begin{figure}[htp]
\begin{center}
\includegraphics[scale=0.4]{kq/vinh_long.png}
\end{center}
\caption{Ký hiệu "Vĩnh Long"}
\end{figure}
\FloatBarrier

\end{itemize}
\thispagestyle{phuluc}
\newpage


%\thispagestyle{phuluc}
\fontsize{12pt}{12pt}{\selectfont{
\addcontentsline{toc}{chapter}{TÀI LIỆU THAM KHẢO}
\bibliographystyle{ieeetr}
\bibliography{references/references}{}
}}
\end{document}